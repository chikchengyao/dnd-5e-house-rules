\documentclass[House_Rules.tex]{subfiles}
\begin{document}

\chapter{Character Options}

This section covers more specific changes that are less universally relevant. 

Reading this section in detail is not necessary, but you should glance through and take note of the ones relevant to your character.




%%%%%%%%%%%%%%%%%%%%%%%%%%%%%%%%
% RACES
%%%%%%%%%%%%%%%%%%%%%%%%%%%%%%%%
\section{Races}

\subsubsection{Ability Score Increase}
If your race and subrace together give a +2 and +1 increase to two fixed ability scores, you may choose which receives the +2 bonus.

You can ignore any reduction to an ability score from a racial Ability Score Increase trait.

\begin{DndComment}{Examples}
High Elves can choose to start with +2 Dexterity and +1 Intelligence, or +1 Dexterity and +2 Intelligence. 

Humans, Mountain Dwarves, and Tritons gain no flexibility because they do not follow the +2/+1 format.

Changelings and Warforged gain no flexibility because their increases are not to fixed ability scores.
\end{DndComment}


\subsubsection{Dragonmarks} You should not use dragonmarked races.

\subsubsection{Flight} Before 5th level, flying speeds granted by racial traits cannot hold you in the air between turns. Unless you have other means of staying aloft, you fall at the end of your turn.

\subsection{Dragonborn}
\textit{Amends Breath Weapon (PH, 34).}

\subparagraph{Breath Weapon} Your breath weapon uses a bonus action. Its damage is 2d4 at 1st level, 3d4 at 6th level, 4d4 at 11th level, and 5d4 at 16th level.

\subsection{Human}

You gain these traits in addition to your base traits (PH, 31). You should not use variant humans.

\subparagraph{Versatile} You gain proficiency in one skill, one tool, one simple weapon, and one martial weapon of your choice.
\subparagraph{Talented} You can double your proficiency bonus in one skill from your class list in which you are proficient.

\subsection{Kobold}
\textit{Replaces Ability Score Increase (VGM, 119).}

\subparagraph{Ability Score Increase} Your Dexterity score increases by 2, and choose one of Constitution, Intelligence, Wisdom or Charisma to increase by 1.

\subsection{Grung}
\textit{Amends Poisonous Skin (OGA, 4).}

\subparagraph{Poisonous Skin} The DC of both types of poison is 10 plus your proficiency bonus.




%%%%%%%%%%%%%%%%%%%%%%%%%%%%%%%%%%%%%%%%%%%
% CLASSES
%%%%%%%%%%%%%%%%%%%%%%%%%%%%%%%%%%%%%%%%%%%
\section{Classes}

\subsection{Multiclassing}
When multiclassing into Barbarian, Paladin or Ranger, you may use Strength in place of Dexterity and vice versa.

You cannot multiclass with a Hexblade Warlock.

\subsection{Monk}

\subsubsection{Martial Arts}
\textit{Amends Monk table, Martial Arts column (PH, 77).}

Your Martial Arts die starts as 1d4. It increases to 1d6 at 5th level, 1d8 at 9th level, 1d10 at 13th level, and 1d12 at 17th level.



\subsection{Ranger}

You may choose one of these two variations:

\begin{itemize}
    \item Original Ranger (PH, 89)
    \item Revised Ranger (UARR, 1)

\end{itemize}

\subsection{Sorcerer}

\subsubsection{Font of Magic}
\textit{Amends Font of Magic (PH, 101).}

You spend sorcery points to \textit{regain} one expended spell slot as a bonus action on your turn (rather than \textit{creating} one).

\subsubsection{Extended Spell}
\textit{Amends Extended Spell (PH, 102).}

You can also extend a spell after it has been cast, but before it has ended. For stacking purposes, it counts as if used when cast.

\subsubsection{Bonus Metamagic}
At 3rd level, you can learn either the Distant Spell or Extended Spell Metamagic options (PH, 102), in addition to the two you normally choose at this level.

\subsubsection{Wild Magic Surge}
\textit{Amends Wild Magic Surge table (PH, 104).}

\begin{DndTable}[]{lX}
    \textbf{d100} & \textbf{Effect} \\
    27-28 & For the next minute, all your spells with a casting time of 1 action can also be cast using 1 bonus action.
\end{DndTable}

\subsection{Warlock}

You can use Intelligence-based warlocks. If you do, all your warlock features based on Charisma, including your saving throw proficiency, change to use Intelligence instead.

Intelligence and Charisma are both considered warlock spellcasting abilities for the purposes of features like the Magic Initiate feat.

\subsection{Wizard}

\subsubsection{Portent}
\textit{Amends Portent (PH, 116).}

After you have used your Portent feature to replace a roll, that roll can no longer be rerolled or replaced by any feature, including Portent.




%%%%%%%%%%%%%%%%%%%%%%%%%%%%%%%%%
% FEATS
%%%%%%%%%%%%%%%%%%%%%%%%%%%%%%%%%
\section{Feats}

\subsection{Changes to Prerequisites}

\paragraph{Grappler} (PH, 167) You no longer need a Strength score of 13 or higher.

\paragraph{Ritual Caster} (PH, 169) You no longer need an Intelligence or Wisdom score of 13 or higher.

\paragraph{Magic Initiate, Ritual Caster, Spell Sniper} (PH, 168-170) You need 11 points or more in the spellcasting ability you choose.

\subsection{GWM and Sharpshooter}
\textit{Amends Great Weapon Master (PH, 167) and Sharpshooter (PH, 170).}

The final benefits of these feats instead offer the option of a penalty equal to your proficiency bonus to attacks, in exchange for adding twice your proficiency bonus to the damage.

\subsection{Lucky}
\textit{Amends Lucky (PH, 167).}

When you use a luck point, roll one additional d20. You can choose to replace one d20 from the original roll with this new d20 roll.




\section{Spells}

\subsection{Identifying Spells}
Once per turn, you can use your reaction to perform a DC 10 + spell level Intelligence (Arcana) check to identify a spell being cast. The DC is 15 + spell level if the spell is not on one of your class lists. You must have perceived a verbal or somatic component, or a material component not substituted by an arcane focus. For each additional such component you perceive, you gain a +2 bonus to your check.

You can choose to cast Counterspell as part of the same reaction.

\subsection{Counterspell}
You gain a +1 to the ability check for each slot level above 3rd.

\subsection{Dispel Magic}
You gain a +1 to the ability check for each slot level above 3rd.




\section{Miscellaneous}

\subsubsection{Telepathy}
Speaking with a creature telepathically does not confer the ability to reply telepathically. This has precedent in Sage Advice (SA2.4, 6).

A creature receiving telepathic speech is aware that the speech is telepathic.

\end{document}