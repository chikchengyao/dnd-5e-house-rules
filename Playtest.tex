\documentclass[House_Rules.tex]{subfiles}
\begin{document}

\chapter{Playtest}
This material can be used but may require further changes. If deemed balanced, features here may be added to the house rules.

\section{Sources}

\begin{DndTable}[header=Sourcebooks: Partial]{Xl}
    \textbf{Sourcebook} & \textbf{Sections} \\
    Explorer's Guide to Wildemount (EGW) & Backgrounds, spells, subclasses \\
\end{DndTable}

\begin{DndTable}[header=Unearthed Arcana]{lX}
    \textbf{Document} & \textbf{Sections} \\
    Class Feature Variants (UACV) & Ranger only (see pg.\pageref{variantRanger}) \\
\end{DndTable}

\section{Characters}

\subsection{Hit Points, Revised}

\subsubsection{Hit Dice}
You gain 2 hit dice per class level. 

\subsubsection{Short Rests}
To gain the benefit of a short rest, you must expend at least 1 hit die, whether or not you regain any hit points.

\subsubsection{Long Rests}
Long rests no longer automatically restore hit points.

At the end of a long rest, you can choose to spend hit dice like in a short rest. Whether or not you do, you then recover half your expended hit dice.




\section{Races}

\subsection{Dragonborn}
\textit{Amends Breath Weapon (PH, 34).}

\subparagraph{Breath Weapon} Your breath weapon uses a bonus action. Its damage is 1d6 at 1st level, 2d6 at 6th level, 3d6 at 11th level, and 4d6 at 16th level.

\subsection{Human}

You gain these traits in addition to your base traits (PH, 31). You should not use variant humans.

\subparagraph{Versatile} You gain proficiency in one skill, one tool, one simple weapon, and one martial weapon of your choice.
\subparagraph{Talented} You can double your proficiency bonus in one skill from your class list in which you are proficient.




\section{Classes}

\subsection{Barbarian}

\subsubsection{Frenzy}
\textit{Replaces Beserker's Frenzy (PH, 49).}

Starting when you choose this path at 3rd level, you can go into a frenzy when you rage. If you do so, for the duration of your rage you can attack one additional time whenever you take the Attack action on your turn. When your rage ends, you suffer one level of exhaustion.

\subsection{Monk}
\subsubsection{Stunning Strike}
\textit{Replaces Stunning Strike (PH, 79).}

Starting at 5th level, you can expend one use of this feature on your turn to select a creature you can see. Until the end of your turn, all your melee weapon attacks against that creature are stunning strikes. If a stunning strike hits a creature, it must succeed on a Constitution saving throw or be stunned until the end of your next turn. You can use this feature a number of times equal to your Wisdom modifier (minimum of 1). When you finish a long rest, you regain all expended uses.

\subsection{Ranger}
You may use a Variant Ranger instead of the other variations.

A Variant Ranger is an Original Ranger amended with these features from Unearthed Arcana (UACV, 7-9):

\begin{itemize}
    \item Deft Explorer
    \item Favored Foe
    \item Fighting Style Options
    \item Ranger Spells
    \item Spellcasting Focus
    \item Primal Awareness
    \item Fade Away
    \item Ranger Companion Options
\end{itemize}

In particular, you should not use the Spell Versatility feature (UACV, 8).

You can also use the new Fighting Styles and the Martial Versatility feature (UACV, 12), but only through Variant Ranger class features.

\subsection{Warlock}

\subsubsection{Hexblade}
The Hex Warrior feature (XGE, 55) is removed from the Hexblade Patron and added to the Pact of the Blade (PH, 107).

You can multiclass with such a Hexblade Warlock.


\section{Combat}

\subsection{Condition: Flanked}
A creature is flanked when at least two hostiles within 5 feet of it overwhelm its ability to defend itself, by virtue of their positioning and/or number, as determined by the DM. Only hostiles a creature is aware of can contribute to flanking it.

Unflanked creatures have advantage on attacks against a flanked creature.

\begin{DndComment}{Example}
A Medium-sized humanoid will usually be flanked by two hostiles standing on opposite sides of it.

However, creatures which have a wide field of view, can easily handle groups of hostiles, or are simply larger, may be harder to flank. In such cases, better positioning or additional hostiles may be required, as determined by the DM.
\end{DndComment}

\subsection{Condition: Prone}
Standing from prone provokes an opportunity attack.


\end{document}