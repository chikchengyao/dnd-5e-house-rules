\documentclass[House_Rules.tex]{subfiles}
\begin{document}

\chapter{Core Rules}

This section covers rules relevant to all characters. This is the only section that you \textit{must} read.

%%%%%%%%%%%%%%%%%%%%%%%%%%%%%%%%%
% SOURCES
%%%%%%%%%%%%%%%%%%%%%%%%%%%%%%%%%
\section{Sources}

\begin{DndTable}[]{Xl}
    \textbf{Sourcebook} & \textbf{Sections} \\
    Dungeon Master's Guide (DMG) & All \\
    Eberron - Rising from the Last War (ERLW) & All \\
    Elemental Evil Player's Companion (EEPC) & All \\
    Guildmaster's Guide to Ravnica (GGR) & Races, subclasses \\
    Monster Manual (MM) & All \\
    Mordenkainen's Tome of Foes (MTF) & All \\
    One Grung Above (OGA) & All \\
    Player's Handbook (PH) & All \\
    Sage Advice Version 2.4 (SA2.4) & All \\
    Sword Coast Adventurer's Guide (SCAG) & All \\
    Volo's Guide to Monsters (VGM) & All \\
    Xanathar's Guide to Everything (XGE) & All \\
\end{DndTable}

\begin{DndTable}[]{lX}
    \textbf{Unearthed Arcana} & \textbf{Sections} \\
    The Ranger, Revised (UARR) & All \\
\end{DndTable}

The errata (if any) for all sourcebooks is specified in Sage Advice (SA2.4, 1).

Sage Advice, including errata, has precedence over all other sources. Unofficial rulings by Jeremy Crawford are used as guidelines.


%%%%%%%%%%%%%%%%%%%%%%%%%%%%%%%%%
% General
%%%%%%%%%%%%%%%%%%%%%%%%%%%%%%%%%
\section{General}
You should use only 27-point buy (PH, 13) or one of the standard arrays listed below, to determine your base ability scores.

\begin{DndTable}[]{lcccccc}
    \textbf{Default} & 8 & 10 & 12 & 13 & 14 & 15 \\
    \textbf{Wide} & 6 & 8 & 12 & 13 & 15 & 16
\end{DndTable}

\subsubsection{Hit Points}
Whenever you roll a natural 1 to increase your maximum hit points upon levelling up, reroll until it is not a natural 1.

\subsubsection{Proficiency Bonus}
You can use the optional rules for proficiency dice (DMG, 263).




%%%%%%%%%%%%%%%%%%%%%%%%%%%%%%%%%
% CLASSES
%%%%%%%%%%%%%%%%%%%%%%%%%%%%%%%%%
\section{Classes}
Multiclassing (PH, 163) can be used.




%%%%%%%%%%%%%%%%%%%%%%%%%%%%%%%%%
% FEATS
%%%%%%%%%%%%%%%%%%%%%%%%%%%%%%%%%
\section{Feats}
Feats (PH, 165) can be used.

\subsection{Spellcasting Feats}

\subsubsection{Minimum Level}
When you take a feat that grants the ability to cast a spell, you cannot cast it using the feat until you reach the level at which a full caster would normally be able to cast the spell.

\subsubsection{Learning Spells}
Ignore any wording in feats that says you learn a spell. This overrules Sage Advice (SA2.4, 8).

\begin{DndComment}{Example}
If you take the Magic Initiate feat, you can cast the chosen 1st level spell once per long rest, but you do not learn that spell, and consequently cannot cast it using spell slots.
\end{DndComment}

\subsection{Bonus Feats}
At character levels 1, 4, 8, 12, 16 and 19, you can take a feat from the Bonus Feats (Single) table (pg.\pageref{bonusFeatsSingleTable}), or a pair of feats from the Bonus Feats (Pair) table (pg.\pageref{bonusFeatsPairTable}). 

You do not benefit from ability score increases from feats taken in this way.

\begin{figure}[htbp]
\label{bonusFeatsSingleTable}
\begin{DndTable}[header=Bonus Feats (Single)]{ll}
    \textbf{Feat} & \textbf{Prerequisite} \\
    Actor & - \\
    Bountiful Luck & Halfling \\
    Charger & - \\
    Drow High Magic & Drow \\
    Dungeon Delver & - \\
    Durable & - \\
    Elemental Adept & Spells \\
    Fade Away & Gnome \\
    Fey Teleportation & High Elf \\
    Healer & - \\
    Keen Mind & - \\
    Mage Slayer & - \\
    Magic Initiate & Casting ability 11+ \\
    Martial Adept & - \\
    Mounted Combatant & - \\
    Orcish Fury & Half-Orc \\
    Prodigy & Human / Half-Elf / Half-Orc \\
    Ritual Caster & Casting ability 11+ \\
    Second Chance & Halfling \\
    Skilled & - \\
    Skulker & Dexterity 13+ \\
    Spell Sniper & Spells, Casting ability 11+ \\
    Squat Nimbleness & Dwarf or Small race \\
    Svirfneblin Magic & Deep Gnome \\
    Wood Elf Magic & Wood Elf
\end{DndTable}

\label{bonusFeatsPairTable}
\begin{DndTable}[header=Bonus Feats (Pair)]{lll}
    \textbf{Feat A} & \textbf{Feat B} & \textbf{Prerequisite} \\
    Athlete & Weapon Master & - \\
    Dragon Fear & Dragon Hide & Dragonborn \\
    Durable & Dwarven Fortitude & Dwarf \\
    Flames of Phlegethos & Infernal Constitution & Tiefling \\
    Grappler & Tavern Brawler & - \\
    Linguist & Observant & -
\end{DndTable}
\end{figure}




%%%%%%%%%%%%%%%%%%%%%%%%%%%%%%%%%
% COMBAT
%%%%%%%%%%%%%%%%%%%%%%%%%%%%%%%%%

\begin{figure}[htbp]
\label{actionsInCombatTable}
\begin{DndTable}[header=Actions in Combat]{ll}
\textbf{Action} & \textbf{Source} \\
Attack & PH, 192 \\
Cast a Spell & PH, 192 \\
Climb onto a Bigger Creature & DMG, 271 \\
Dash & PH, 192 \\
Disarm & DMG, 271 \\
Disengage & PH, 192 \\
Dodge & PH, 192 \\
Help & PH, 192 \\
Hide & PH, 192 \\
Interact with an Object & PH, 190 \\
Overrun & DMG, 272 \\
Ready & PH, 193 \\
Search & PH, 193 \\
Shove Aside & DMG, 272 \\
Tumble & DMG, 272 \\
Use a Bonus Action & pg.\pageref{useABonusAction} \\
Use an Object & PH, 193 \\
\end{DndTable}
\end{figure}

\section{Combat}

\subsection{Actions}
Standard actions are exhaustively listed in the Actions in Combat table (pg.\pageref{actionsInCombatTable}). In particular, you should not use the Mark action (DMG, 271). You may also use an action from a feature or item you possess, or improvise an action.


\subsection{Action: Use a Bonus Action}
\label{useABonusAction}
You can use your action to use a bonus action, but you cannot use bonus actions from the same feature more than once in a turn. This overrules Sage Advice (SA2.4, 10).

The modified rules on bonus action casting (pg.\pageref{bonusActionSpells}) also apply to spells cast using this action.

\begin{DndComment}{Example}
You can use your action to cast Healing Word with a bonus action, and then use your regular bonus action to attack with a Spiritual Weapon. 

However, you may not attack twice with a Spiritual Weapon because this uses two bonus actions from the same spell, even if you attack with two separate castings of the spell, or if one bonus action was used to attack with an existing Spiritual Weapon while the second bonus action was used to cast a new Spiritual Weapon.
\end{DndComment}

\subsection{Cover}
Ranged attacks through Medium-sized or smaller openings have half or three-quarters cover, as determined by the DM. 

An attacker does not suffer this penalty if it is within 5 feet of the opening.

\subsection{Falling Unconscious}
If damage reduces you to 0 hit points and fails to kill you, you fall unconscious \textit{and gain one level of exhaustion}.


%%%%%%%%%%%%%%%%%%%%%%%%%%%%%%%%%
% SPELLCASTING
%%%%%%%%%%%%%%%%%%%%%%%%%%%%%%%%%
\section{Spellcasting}

\subsubsection{Bonus Action Spells}
\label{bonusActionSpells}
\textit{Replaces Casting Time: Bonus Action (PH, 202).}

If you use a bonus action to cast a spell, you can't cast another spell during the same turn, except for a cantrip with a casting time of 1 action.

\subsubsection{Targeting}
Spells that target creatures can also target objects where reasonable, as determined by the DM. This overrules Sage Advice (SA2.4, 13).




%%%%%%%%%%%%%%%%%%%%%%%%%%%%%%%%
% MISCELLANEOUS
%%%%%%%%%%%%%%%%%%%%%%%%%%%%%%%%
\section{Miscellaneous}

\subsubsection{Critical Failure}
A critical failure occurs when you have disadvantage on an attack roll or ability check and roll two natural 1s. In addition to missing or failing the check, something else happens as determined by the DM. 

\subsubsection{Diagonal Grid Movement}

Every second consecutive diagonal you move costs an additional 5 feet of movement.

Additionally, you have width while moving diagonally.

\begin{DndComment}{Example}
Because you have width, if you take a step northeast you will also pass through the north and the east grid squares. Therefore, you cannot move diagonally if either of those squares are blocked, and you provoke opportunity attacks from both squares.
\end{DndComment}

\subsubsection{Help}
You can help with an ability check that uses a proficiency only if you have the relevant proficiency.

\subsubsection{Improvised Weapons}
Improvised weapons have the statistics of the most similar \textit{simple} weapon, as determined by the DM.

An object used as an improvised weapon becomes unusable as a weapon after one combat.

\subsubsection{Instant Advantage}
Features which instantly grant advantage on an attack roll, ability check or saving throw can also be used to reroll one d20 from the original roll, after seeing a roll but before knowing the outcome.


\end{document}