\documentclass[House_Rules.tex]{subfiles}
\begin{document}

\chapter{Playtest Rules}
These are proposed changes to the core rules. They may be used on a session-by-session basis for playtesting purposes. 

After sufficient playtesting, features here may be added to the core rules.

\section{Characters}

\subsection{Revised Hit Points}

\subsubsection{Hit Dice}
You gain 2 hit dice per class level. 

\subsubsection{Short Rests}
To gain the benefit of a short rest, you must expend at least 1 hit die, whether or not you regain any hit points.

\subsubsection{Long Rests}
Long rests no longer automatically restore hit points.

At the end of a long rest, you can choose to spend hit dice like in a short rest. Whether or not you do, you then recover half your expended hit dice, rounded up.

\section{Combat}

\subsection{Flanking}
A flanked creature suffers a -2 penalty to AC.

A creature is flanked if there are two hostiles within 5 feet of it, where one hostile is occupying a grid square directly opposite another hostile, or is adjacent to such a square.

\subsubsection{Immunity}
Certain creatures, such as beholders, are immune, as well as:
\begin{itemize}
    \item creatures within 5 feet of an ally
    \item creatures whose AC does not use their Dexterity modifier
    \item creatures of size Huge or larger
\end{itemize}

\begin{DndComment}{Example}
If a non-immune creature has a hostile 5 feet to its north, it will be flanked if there is also a hostile 5 feet to its southwest, south, or southeast.
\end{DndComment}

\subsection{High Ground}
Ranged weapon attacks against a target at least 5 feet above the attacker suffer a -2 penalty to hit.

Similarly, ranged weapon attacks against a target at least 5 feet below have a +2 bonus to hit.

\subsection{Potions}

Potions can be drunk as a bonus action. Administering potions to another creature still requires a full action.

\subsection{Prone}
Standing from prone provokes an opportunity attack.


\end{document}