% bg = [ full | none | print ]
\documentclass[letterpaper,twocolumn,openany,nodeprecatedcode,bg=print]{dndbook}

% Use babel or polyglossia to automatically redefine macros for terms
% Armor Class, Level, etc...
% Default output is in English; captions are located in lib/dndstrings.sty.
% If no captions exist for a language, English will be used.
%1. To load a language with babel:
%	\usepackage[<lang>]{babel}
%2. To load a language with polyglossia:
%	\usepackage{polyglossia}
%	\setdefaultlanguage{<lang>}
\usepackage[english]{babel}
%\usepackage[italian]{babel}
% For further options (multilanguage documents, hypenations, language environments...)
% please refer to babel/polyglossia's documentation.

\usepackage{appendix}
\usepackage[singlelinecheck=false]{caption}
\usepackage[colorlinks = true,
            linkcolor = blue,
            urlcolor  = blue,
            citecolor = blue,
            anchorcolor = blue]{hyperref}
\usepackage[utf8]{inputenc}
\usepackage{lipsum}
\usepackage{listings}
\usepackage{shortvrb}
\usepackage{stfloats}
\usepackage{subfiles}

\MakeShortVerb{|}

\newcommand{\pg}[1]{page \pageref{#1}}
\newcommand{\see}[1]{(see \pg{#1})}

\lstset{%
  basicstyle=\ttfamily,
  language=[LaTeX]{TeX},
  breaklines=true,
}

\setcounter{tocdepth}{2}

\raggedbottom

\begin{document}

%\tableofcontents




\chapter{Core Houserules}

\noindent [Version 7.6.1]

\section{How To Use}
The rules on this page are mandatory. Every other rule in this document is \textit{optional.} 

\subsubsection{Optional Rules}
Some optional rules provide mild benefits, while others can be self-imposed challenges. None are overly powerful or detrimental. 

All optional rules are designed such that choosing it affects only you, and no one else. So, you are free to use any combination of optional rules, or none of them.

\subsubsection{Sources and Balancing}
You can use any material from the official sourcebooks \see{sources}. Ask before using Unearthed Arcana or homebrew.

Balancing will be done on a case-by-case basis, when it seems necessary. 

\subsubsection{New Players}
A primer to D\&D 5e can be found in Appendix \ref{new-players-guide} (\pg{new-players-guide}).

\newpage
\section{Houserules}

\subsection{Base Ability Scores}
Use standard \href{https://chicken-dinner.com/5e/5e-point-buy.html}{27-point buy} (PH, 13) for your ability scores. 

At 1st level, ability scores are capped at 17, after racial bonuses.

\subsection{Difficulty Classes}
To succeed, a roll must strictly exceed, not just equal, its DC. This includes attacks against AC. Reasons are explained in Appendix \ref{difficulty-classes} (\pg{difficulty-classes}).

\subsection{Falling Unconscious}
If you fall unconscious from damage, you gain one level of exhaustion.

\subsection{Prone}
Standing from prone provokes opportunity attacks, which occur just before standing. Taking the Disengage action avoids them.

\subsection{Counterspell}
The spell "Counterspell" no longer exists. An optional alternative is available \see{counterspell}.

\subsection{Bonus Action Spells}
The rule on Bonus Action Spells (PH, 202) is simplified to: you can cast at most one levelled spell per turn.

\subsection{Warlocks}
Warlocks are based on Intelligence, not Charisma. Also, the Hexblade's Hex Warrior feature (XGE, 55) is transferred to the Pact of the Blade (PH, 107).










\chapter{Character Options}
\noindent You can opt to use any number of these for your own character.

\subsection{Simple HP Formula}
You can opt to use this formula, which is never lower than the standard formula.
\begin{itemize}
\item Your HP without class levels is 5.
\item For every class level, your HP increases by one hit die roll + CON mod. Reroll 1s. 
\item Instead of rolling, you can take the average, rounded up.
\end{itemize}

\subsection{Intelligence Proficiencies}
For every +1 in your base Intelligence modifier (minimum 0), you can opt to learn one language and one of the following:
\begin{itemize}
    \item learn one tool, or
    \item learn one Int, Wis or Cha skill, or
    \item double your proficiency in an Int skill.
\end{itemize}



\subsection{Races}
You may swap your race's ability score increase for a +2/+1 increase to two different ability scores.

\subsection{Multiclassing}
Multiclassing (PH, 163) is allowed. 

When multiclassing into or from a Barbarian, Paladin or Ranger, you may use Strength instead of Dexterity (and vice versa) to fulfil the ability score minimum. 

\subsection{Tool Proficiencies}
When you take your first level of a martial class, and every time you gain an Ability Score Increase from it, you can gain a tool proficiency or upgrade an existing tool proficiency to expertise. 

\subsection{Backgrounds}
You can use custom backgrounds. Choose:
\begin{itemize}
\item any two skill proficiencies 
\item a total of two tool proficiencies or languages 
\item one equipment package from any background
\item one background feature from any background
\end{itemize}

\subsection{Feats}
Feats (PH, 165) are allowed, except Lucky and Savage Attacker.

\subsubsection{New Prerequisites}

\paragraph{Dwarven Fortitude} (XGE, 74) You no longer need to be a Dwarf.

\paragraph{Grappler} (PH, 167) You no longer need a Strength score of 13 or higher.

\paragraph{Ritual Caster} (PH, 169) You no longer need an Intelligence or Wisdom score of 13 or higher.

\paragraph{Magic Initiate, Ritual Caster, Spell Sniper} (PH, 168-170) You need 11 points or more in the spellcasting ability you choose.

\subsection{Flavour Feats}
\label{flavour-feats}
At level 1, and whenever you gain an Ability Score Increase from your class, you can opt to take one flavour feat from the table in the Appendix (\pg{flavour-feats-table}). Ignore ability score increases from flavour feats.

Flavour feats are not strong, but have good roleplaying potential and provide situational benefits.

\subsection{Counterspell}
\label{counterspell}
By the core houserules, the spell "Counterspell" no longer exists. If you are a bard, or a spellcaster who would originally have access to Counterspell, you can opt to gain this feature at the class level where you gain 3rd-level spells:

\subsubsection{Counterspell}
If you perceive a spell being cast, you automatically identify it, and can attempt to disrupt that triggering spell based on your knowledge of its school of magic, before it takes effect.

A counterspell is a spell from the same school as the triggering spell, which you know or have prepared. You cast your counterspell while adjusting it to disrupt the triggering spell, instead of producing its usual effect. Your counterspell can be cast using a reaction instead of its normal casting time, and requires its usual components, but does not consume them.

Basic spells cannot counteract advanced spellwork. Your counterspell cannot be a cantrip, and its base level cannot be lower than the triggering spell's base level.

Make a spellcasting ability check against DC 10 + the triggering spell's level. You gain a bonus to this check equal to the level of the spell slot you used. Note that this check is usually made without proficiency.

If your counterspell is the same spell as the triggering spell, you gain advantage on this check, as less adjustment is needed. If your counterspell is cast at a lower level than the triggering spell, you have disadvantage on this check.

\subsection{Vancian Magic}
Prepared spellcasters can opt to be Vancian spellcasters, exchanging some spellcasting flexibility for limited access to their entire spell list.

\subsubsection{Preparing Spells}
Instead of preparing a list of spells, you prepare spells inside each of your spell slots. Whenever you complete a long rest or otherwise recover spell slots, choose a spell to occupy each slot. You can use a spell slot only to cast the spell occupying it, unless you take the Reprepare action (below) to change the occupying spell.

\begin{DndComment}{Example}
After completing a long rest, a 3rd-level Cleric who uses Vancian magic might have these spells prepared:

\begin{itemize}
\item \textit{Bless}, \textit{Guiding Bolt}, and two \textit{Cure Wounds}, at 1st level.
\item \textit{Cure Wounds} and \textit{Inflict Wounds}, at 2nd level.
\end{itemize}

\noindent This cleric can cast Bless at 1st level, expending that spell. After doing so, her remaining prepared spells are:

\begin{itemize}
\item \textit{Guiding Bolt}, and two \textit{Cure Wounds}, at 1st level.
\item \textit{Cure Wounds} and \textit{Inflict Wounds}, at 2nd level.
\end{itemize}

\noindent Now, she can no longer cast Bless because she no longer has it prepared. 

Because Vancian casters can cast spells only at the level they were prepared, she can cast \textit{Guiding Bolt} only at 1st level, and \textit{Inflict Wounds} only at 2nd level.
\end{DndComment}

\subsubsection{Action: Reprepare}
You can use the Reprepare action a number of times equal to your spellcaster level, recovering all expended uses on a long rest. You can reprepare a spell occupying a spell slot by taking the Reprepare action for a number of consecutive rounds equal to the slot's level. Repreparation is completed at the start of your turn on the round after your final Reprepare action. 

You choose the slot to reprepare and the new spell to occupy it when you take the first Reprepare action. You must concentrate on Reprepare until the repreparation is complete. If you do not take enough consecutive Reprepare actions, or your concentration is broken, the repreparation fails, wasting any expended uses of Reprepare.

If a feature says you have a spell "always prepared", using Reprepare to convert another spell to that spell does not count against your limit.

%\begin{DndTable}[header=Vancian Spell Slots]{r | ccccccccc}
%\textbf{Level} & \textbf{1st} & \textbf{2nd} & \textbf{3rd} & \textbf{4th} & \textbf{5th} & \textbf{6th} & \textbf{7th} & \textbf{8th} & \textbf{9th} \\
%\hline
%1st & 4 & - & - & - & - & - & - & - & - \\
%2nd & 5 & - & - & - & - & - & - & - & - \\
%3rd & 5 & 3 & - & - & - & - & - & - & - \\
%4th & 6 & 4 & - & - & - & - & - & - & - \\
%5th & 6 & 4 & 3 & - & - & - & - & - & - \\
%6th & 6 & 4 & 4 & - & - & - & - & - & - \\
%7th & 6 & 4 & 4 & 2 & - & - & - & - & - \\
%8th & 6 & 5 & 4 & 3 & - & - & - & - & - \\
%9th & 6 & 5 & 4 & 3 & 2 & - & - & - & - \\
%10th & 6 & 5 & 5 & 3 & 3 & - & - & - & - \\
%11th & 6 & 5 & 5 & 3 & 3 & 1 & - & - & - \\
%12th & 6 & 6 & 5 & 4 & 3 & 1 & - & - & - \\
%13th & 6 & 6 & 5 & 4 & 3 & 1 & 1 & - & - \\
%14th & 6 & 6 & 5 & 4 & 3 & 2 & 1 & - & - \\
%15th & 6 & 6 & 5 & 4 & 3 & 2 & 1 & 1 & - \\
%16th & 6 & 6 & 5 & 4 & 3 & 2 & 2 & 1 & - \\
%17th & 6 & 6 & 5 & 4 & 3 & 2 & 2 & 1 & 1 \\
%18th & 6 & 6 & 5 & 4 & 4 & 2 & 2 & 1 & 1 \\
%19th & 6 & 6 & 5 & 4 & 4 & 3 & 2 & 1 & 1 \\
%20th & 6 & 6 & 5 & 4 & 4 & 3 & 2 & 2 & 1 \\
%\end{DndTable}







\chapter{Gameplay Options}
\noindent You can opt to use any number of these to modify your own gameplay.

\subsection{Proficiency Dice}
You can opt to use proficiency dice instead of a flat proficiency bonus (DMG, 263).

\subsection{Hit Dice}
This option gives you more precise control over healing from rests.

\subsubsection{Levelling}
You gain 2 hit dice per class level, instead of 1.

\subsubsection{Long Rests}
Long rests no longer automatically restore hit points. As a long rest ends, you can spend hit dice like in a short rest. Then, recover half your expended hit dice, rounded up.

\subsection{Encumbrance}
You can opt to track encumbrance as a self-imposed challenge. Use any of the three modes below.

\begin{DndTable}[header=Encumbrance (Normal)]{rX}
\textbf{Carried weight} & \textbf{Effect} \\
$\leq$ 15 $\times$ Strength & No penalty \\
$>$ 15 $\times$ Strength & Physically impossible \\
\end{DndTable}

\begin{DndTable}[header=Encumbrance (Hard)]{rX}
\textbf{Carried weight} & \textbf{Effect} \\
$\leq$ 5 $\times$ Strength & No penalty \\
$>$ 5 $\times$ Strength & Cannot travel at a fast pace \\
$>$ 10 $\times$ Strength & All speeds -10ft. \\
$>$ 15 $\times$ Strength & Physically impossible \\
\end{DndTable}

\begin{DndTable}[header=Encumbrance (Expert)]{rX}
\textbf{Carried weight} & \textbf{Effect} \\
$\leq$ 5 $\times$ Strength & No penalty \\
$>$ 5 $\times$ Strength & All speeds -10ft. \\
$>$ 10 $\times$ Strength & All speeds -20ft., disadvantage on physical checks, attacks and saves. \\
$>$ 15 $\times$ Strength & Physically impossible \\
\end{DndTable}
\subsection{Diagonal Grid Movement}
You can choose to move more realistically on a grid. Every second diagonal on a turn costs double (DMG, 252). Additionally, you have width while moving diagonally. 

\begin{DndComment}{Example}
Because you have width, if you step northeast you will also pass through the north and east grid squares. Therefore, you can't step diagonally if either of those squares are blocked, and you provoke opportunity attacks from both squares.
\end{DndComment}

\subsection{Critical Hits}
\label{critical-hit}
Whenever you make a critical hit, choose one of the following:
\begin{itemize}
\item maximise the dice (no uncertainty)
\item roll the dice twice (default)
\item roll once and double (high risk, high reward)
\end{itemize}

\subsection{Action: Gain a Bonus Action}
\label{gameplay-bonus-action}
You can exchange an action for a bonus action on your turn, but you cannot use bonus actions from the same feature more than once in a turn.

\begin{DndComment}{Example}
You can use your action to gain a second bonus action, then cast Healing Word with one bonus action and use your remaining bonus action to attack with a Spiritual Weapon. 

However, you may not attack twice with a Spiritual Weapon because this uses two bonus actions from the same spell, even if you attack with two separate castings of the spell, or if one bonus action was used to attack with an existing Spiritual Weapon while the other was used to cast a new Spiritual Weapon.
\end{DndComment}

\subsection{Combat Options}
Besides the standard options like Dash and Shove, you can use Disarm, Shove Aside, Climb Onto, Overrun and Tumble (DMG, 271-272). 

For your convenience, a reference list of all combat options is provided in the Appendix \see{combat-options}.















\chapter{Clarifications}

\noindent These clarifications must be adhered to, but are rare enough in practice that \textit{reading} them is not mandatory.

\subsection{Feat Preclusions}
You cannot be a Fighter (Battlemaster) and have the Martial Adept feat.

You cannot have levels in the same class in which you have the Magic Initiate feat.

If such a conflict arises, you can change the conflicting feat.

\subsection{Spellcasting Feats}

\subsubsection{Minimum Level}
Before you can cast a spell granted by a feat, you must first reach the level at which a full caster can cast that level of spell.

\subsubsection{Learning Spells}
Ignore any wording in feats that says you "learn" a spell.

\begin{DndComment}{Example}
If you take the Magic Initiate feat, you can cast the chosen 1st level spell once per long rest, but you do not learn that spell, and therefore cannot cast it using spell slots.
\end{DndComment}










\appendix
\chapter{Tables}

\label{sources}
\begin{DndTable}[header=\href{https://thetrove.is/Books/Dungeons\%20\%26\%20Dragons\%20\%5Bmulti\%5D/5th\%20Edition\%20\%285e\%29/Core/}{Official Sourcebooks}]{rl}
\textbf{Abbv.} & \textbf{Name} \\
DMG & Dungeon Master's Guide \\
ERLW & Eberron - Rising from the Last War \\
EE & Elemental Evil Player's Companion \\
EGW & Explorer's Guide to Wildemount \\
GGR & Guildmaster's Guide to Ravnica \\
MM & Monster Manual \\
MTF & Mordenkainen's Tome of Foes \\
PH & Player's Handbook \\
SCAG & Sword Coast Adventurer's Guide \\
TCE & Tasha's Cauldron of Everything \\
VGM & Volo's Guide to Monsters \\
XGE & Xanathar's Guide to Everything \\
\end{DndTable}

\label{combat-options}
\begin{DndTable}[header=Standard Combat Options]{llX}
\textbf{Cost} & \textbf{Option} & \textbf{Brief description} \\
Action & Dash & Extra movement equal to your speed. \\
Action & Disengage & Avoid opportunity attacks. \\
Action & Dodge & Until next turn, disadvantage to be hit, and advantage on Dex saves. \\
Action & Help & Give ally advantage on next check or attack. \\
Action & Hide & - \\
Action & Ready & Hold any other action until triggered. \\
Bonus & Improvise & Do something nonstandard, at DM's discretion. \\
Bonus & Search & Attempt to find a hidden object or creature. \\
1 attack & Grapple & Attempt to grab opponent (+1 size). (Ath vs. Ath/Acr) \\
1 attack & Shove & Attempt to push 5ft. away or knock prone (+1 size). (Ath vs. Ath/Acr)\\
\end{DndTable}

\begin{DndTable}[header=Advanced Combat Options]{llX}
\textbf{Cost} & \textbf{Option} & \textbf{Brief description} \\
Action & Gain B.A. & Gain another bonus action. \\
1 attack & Disarm & Attempt to knock weapon or item from opponent's hands. (Attack vs. Ath/Acr) \\
1 attack & Shove Aside & Attempt to push opponent 5ft. to the side (+1 size). (Ath[disadv] vs. Ath/Acr) \\
Half speed & Climb Onto & Attempt to climb onto creature (+2 size), gaining advantage on attacks against it. You may be thrown off. (Ath vs. Acr) \\
5ft. & Overrun & Attempt to charge through opponent's space. (Ath vs. Ath/Acr) \\
5ft. & Tumble & Attempt to slip through opponent's space. (Acr vs. Ath/Acr) \\
\end{DndTable}

\label{official-races}
\begin{DndTable}[header=Official Races]{lXc}
    \textbf{Race} & \textbf{Subrace(s)} & \textbf{Source} \\
    Aarakocra & - & EEPC, 5 \\
    Aasimar & Fallen, Protector, Scourge & VGM, 104 \\
    Bugbear & - & ERLW, 25 \\
    Centaur & - & GGR, 15 \\
    Changeling & - & ERLW, 18 \\
    Dragonborn & - & PH, 34 \\
    Dwarf & Hill, Mountain & PH, 20 \\
      & Duergar & SCAG, 104 \\
      & Mark of Warding & ERLW, 51 \\
    Elf & Dark/Drow, High, Wood & PH, 23 \\
      & Eladrin, Sea, Shadar-kai & MTF, 62-63 \\
      & Mark of Shadow & ERLW, 49 \\
    Firbolg & - & VGM, 107 \\
    Genasi & Air, Earth, Fire, Water & EEPC, 9 \\
    Gith & Githyanki, Githzerai & MTF, 96 \\
    Gnome & Forest, Rock & PH, 36 \\
      & Deep & SCAG, 115 \\
      & Mark of Scribing & ERLW, 47 \\
    Goblin & - & ERLW, 26 \\
    Goliath & - & VGM, 109 \\
    Half-Elf & Standard & PH, 39 \\
      & Specific elven descent & SCAG, 116 \\
      & Marks of Detection, Storm & ERLW, 40,50 \\
    Half-Orc & - & PH, 41 \\
      & Mark of Finding & ERLW, 41 \\
    Halfling & Lightfoot, Stout & PH, 28 \\
      & Ghostwise & SCAG, 110 \\
      & Marks of Healing, Hospitality & ERLW, 43-44 \\
    Hobgoblin & - & ERLW, 26 \\
    Human & - & PH, 31 \\
      & Marks of Finding, Handling, Making, Passage, Sentinel & ERLW, 41-48 \\
    Kalashtar & - & ERLW, 30 \\
    Kenku & - & VGM, 111 \\
    Kobold & - & VGM, 119 \\
    Lizardfolk & - & VGM, 113 \\
    Loxodon & - & GGR, 18 \\
    Minotaur & - & GGR, 19 \\
    Orc & - & ERLW, 32 \\
    Shifter & Beasthide, Longtooth, Swiftstride, Wildhunt & ERLW, 33-34 \\
    Simic Hybrid & - & GGR, 20 \\
    Tabaxi & - & VGM, 115 \\
    Tiefling & Standard (aka Asmodeus) & PH, 43 \\
      & Devil's Tongue, Feral, Hellfire & SCAG, 118 \\
      & Winged & SCAG, 118 \\
      & Baalzebul, Dispater, Fierna, Glasya, Levistus, Mammon, Mephistopheles, Zariel & MTF, 21-23 \\
    Triton & - & VGM, 117 \\
    Vedalken & - & GGR, 21 \\
    Warforged & - & ERLW, 36 \\
    Yuan-ti & - & VGM, 120 \\
\end{DndTable}

\onecolumn
\section{Flavour Feats}
\label{flavour-feats-table}

You can take a single flavour feat, or a matched pair of feats (e.g. 3A and 3B) from the Flavour Feats (Paired) table. Such a pair counts as a single flavour feat. ASIs do not apply when feats are chosen as flavour feats, but are included for reference in case you choose these feats as normal feats.

\begin{DndTable}[header=Flavour Feats (Single)]{llll}
    \textbf{Feat} & \textbf{Prerequisite} & \textbf{ASI} & \textbf{Benefit} \\
    Actor & - & CHA & Mimic people and sounds. \\
    Bountiful Luck & Halfling & - & Give Lucky to allies. \\
    Charger & - & - & When Dashing, also attack or shove. \\
    Drow High Magic & Elf (Drow) & - & Detect magic, levitate, and dispel magic. \\
    Dungeon Delver & - & - & Find secret doors and resist traps. \\
    Elemental Adept & Spells & - & Strengthen spells of one element. \\
    Fade Away & Gnome & DEX/INT & Turn invisible when damaged. \\
    Fey Teleportation & Elf (High) & INT/CHA & Learn Sylvan and misty step. \\
    Healer & - & - & Nonmagical healing. \\
    Keen Mind & - & INT & Photographic memory. \\
    Mage Slayer & - & - & Harass casters in melee. \\
    Magic Initiate & Casting ability 11+ & - & Two cantrips and one spell from any class. \\
    Martial Adept & - & - & Special moves in combat. \\
    Mounted Combatant & - & - & More accuracy and protect your mount. \\
    Orcish Fury & Half-Orc & STR/CON & Burst damage, and revenge when knocked out. \\
    Prodigy & Human / Half-Elf / Half-Orc & - & One each of language, skill, tool, expertise. \\
    Ritual Caster & Casting ability 11+ & - & Ritual spells from one class. \\
    Second Chance & Halfling & DEX/CON/CHA & Dodge one attack per combat. \\
    Shield Master & - & - & Better DEX saves, shove with shield. \\
    Skilled & - & - & 3 skills or tools. \\
    Skulker & Dexterity 13+ & - & More hiding places, see better in dim light. \\
    Spell Sniper & Spells, Casting ability 11+ & - & Doubles spell attack range, learn one cantrip. \\
    Squat Nimbleness & Dwarf or Small race & STR/DEX & Move faster, learn Ath/Acr, escape grapples.\\
    Svirfneblin Magic & Gnome (Deep) & - & Nondetection, blind/deafness, blur, disguise self. \\
    Wood Elf Magic & Elf (Wood) & - & Druid cantrip, longstrider, pass without trace. \\
\end{DndTable}

\begin{DndTable}[header=Flavour Feats (Paired)]{lllll}
    \textbf{Pair} & \textbf{Feat} & \textbf{Prerequisite} & \textbf{ASI} & \textbf{Benefit} \\
    \hline
    1A & Athlete & - & STR/DEX & Easily climb, jump and stand from prone. \\
    1B & Weapon Master & - & STR/DEX & Four weapon proficiencies. \\
    \hline
    2A & Dragon Fear & Dragonborn & STR/CON/CHA & Use breath weapon to frighten. \\
    2B & Dragon Hide & Dragonborn & STR/CON/CHA & Natural armour and claws. \\
    \hline
    3A & Durable & - & CON & More reliable healing from hit dice. \\
    3B & Dwarven Fortitude & \textbf{none} & CON & Heal while Dodging. \\
    \hline
    4A & Flames of Phlegethos & Tiefling & INT/CHA & Somewhat improve fire spells. \\
    4B & Infernal Constitution & Tiefling & CON & Resist cold and poison. \\
    \hline
    5A & Grappler & - & - & Advantage to hit grapplee, or pin them. \\
    5B & Tavern Brawler & - & STR/CON & Fight and grapple, without weapons. \\
    \hline
    6A & Linguist & - & INT & Learn 3 languages, and create ciphers. \\
    6B & Observant & - & INT/WIS & Read lips, better passive alertness. \\
\end{DndTable}

\twocolumn




\chapter{Difficulty Classes}
\label{difficulty-classes}

Under the core houserules, a roll must exceed its DC to succeed. This includes attack rolls against AC. This change aims to make DCs more intuitive and reversible.

\subsection{Intuitive DCs}

Under the standard rules, an average human with 10 in every ability would have the following success rates on checks:

\begin{itemize}
\item 80\% on a DC5 check,
\item 55\% on a DC10 check,
\item 30\% on a DC15 check.
\end{itemize}

\noindent Under the core houserules, the success rates change to the following, which is easier to intuit:

\begin{itemize}
\item 75\% on a DC5 check,
\item 50\% on a DC10 check,
\item 25\% on a DC15 check.
\end{itemize}

\noindent Additionally, a +7 roll against DC15 has a 60\% chance of succeeding, which is easily intuited as two 5\% steps above 50\%. Contrast this to the 65\% chance under the normal rules.

\subsection{Reversible DCs}
Under the core houserules, an enemy with +7 to attack against a player's AC15 has a 60\% chance of succeeding. From the player's point of view, the player has 40\% chance to defend. 

But notice that a +5 check against DC17 will also have 40\% chance to succeed. In this manner, any attack roll by the DM can be easily flipped around, becoming a defence check rolled by the player.

This paves the way for a system where players roll all the dice, increasing player agency and reducing DM workload.

\section{Drawbacks}
This houserule shifts combat balance somewhat, making attacks weaker and abilities that force a saving throw stronger. 





\chapter{New Player's Guide}
\label{new-players-guide}

% Basics
\newpage
\section{Characters}

All characters have six \textbf{abilities}: Strength, Dexterity, Constitution, Intelligence, Wisdom, and Charisma. They represent how good characters are at different types of tasks. Characters also have features from:

\begin{itemize}
\item their \textbf{race}, like "human" or "elf",
\item their \textbf{class}, like "paladin" or "wizard",
\item their \textbf{background}, like "entertainer" or "spy".
\end{itemize}

\subsection{Abilities}
Your character has a score for each ability. An ability score of 10 is the human average.

Ability scores translate to a \textbf{modifier}. These modifiers are much more important than their underlying scores, because modifiers are added directly to dice rolls. An ability score of 10 corresponds to a modifier of 0. Every 2 extra points in an ability score grants a +1 to your modifier. 

\begin{DndTable}[]{ll}
    \textbf{Ability score} & \textbf{Modifier} \\
    8-9 & –1 \\
    10-11 & ±0 \\
    12-13 & +1 \\
    14-15 & +2 \\
    ... & ... \\
\end{DndTable}

\begin{DndComment}{Intelligence and Wisdom}

\noindent Intelligence (book-smarts) and Wisdom (street-smarts) are sometimes confused. They can often achieve the same goal, but these examples illustrate their differences:
\begin{itemize}
\item Intelligence helps you deduce the best path through a maze, while Wisdom helps you spot hidden shortcuts.
\item Intelligence helps you identify a creature's habitat, while Wisdom helps you track the creature through it.
\item Intelligence helps you solve a sphinx's puzzle, while Wisdom helps you realise it intends to kill you anyway.
\end{itemize}
\end{DndComment}

\subsection{Race, Class, Background}
Your race, class and background give you features that differentiate you from other characters. 

Races represent your genetics and culture of origin. For example, dwarves are short and slow, but have darkvision, can speak Dwarvish, and have learned to wield warhammers. 

Classes represent your actively developed capabilities. For example, a Fighter will continually improve their skill at arms, while a druid gains more powerful spells as she grows more attuned with nature. 

Backgrounds represent your life experience. For example, a spy is good at deception and stealth, and knows how to pick locks. 


% d20 rolls
\newpage
\section{D20 Rolls}

A d20 is a 20-sided die. There are three kinds of d20 rolls: \textbf{checks}, \textbf{attacks}, and \textbf{saves}.

\subsection{Checks}
Ability \textbf{checks}, or just "checks", are attempts to do something difficult, like bluff a guard, search for a hidden door, or cheat at a card game. Checks are tied to an ability. Roll a d20, and add the associated ability modifier. For example, if your Intelligence is 16, your modifier is +3. Therefore, your Intelligence check will be d20 + 3, which is at least 4 and at most 23.

If your check exceeds the task's Difficulty Class, or DC, you succeed.

\begin{DndComment}{Proficiency}

\noindent Special training or experience is represented by \textit{proficiency}. For example, a 1st-level character proficient in History could add their proficiency bonus of +2 to any History-related check. Your proficiency bonus increases slowly as you level up.

\textbf{Proficiency affects only checks, attacks and saves}. You do not add proficiency to damage rolls, for instance.

\end{DndComment}

\subsection{Attacks}
An \textbf{attack} is an attempt to aim a weapon at a small target, such as a gap in an enemy's armour. Roll a d20, and add your Strength modifier for melee weapons or Dexterity modifier for ranged. If you are proficient with the weapon used, add your proficiency bonus.

If your attack roll exceeds the target's Armour Class, or AC, then your attack hits, and you can proceed to make the weapon's damage roll.

A special feature of attacks is that a natural 1 on the d20 always misses, and a natural 20 is a critical hit \see{critical-hit}, doing extra damage.

\subsection{Saves}
A saving throw, also called a "\textbf{save}", is an instinctive, involuntary action to save yourself from an effect. Saves are tied to an ability. For example, you might roll a Dexterity save to avoid a fireball, or a Constitution save to resist poison. Roll a d20, and add the linked ability modifier. If you are proficient in the save of the associated ability, add your proficiency bonus.

If your save exceeds the effect's Difficulty Class, or DC, you succeed.

Proficiency in saves is much more difficult to obtain than checks or attacks. Most characters will only ever have two save proficiencies.





\section{Combat}
Combat occurs in \textbf{rounds} lasting 6 seconds each, so that there are 10 rounds in a minute. During a round, every participant gets a \textbf{turn}. Turn order is determined at the start of combat, by \textit{rolling for initiative}.

\subsection{Actions}
At the start of each of your turns, you get:

\begin{itemize}
\item 1 action
\item 1 bonus action
\item 1 object interaction
\item 1 reaction
\end{itemize}

Your \textbf{action} is the most powerful and flexible of the four. It is usually used to Attack or Cast A Spell, but you can take any actions from the standard list \see{combat-options}, or take an action from your class or race.

Your \textbf{bonus action} can only be used to do specific things which say they cost a bonus action. For example, Barbarians need to use a bonus action to enter a rage, while a Wizard might never use a bonus action. Most characters start with nothing that uses bonus actions.

An \textbf{object interaction} can be used to draw one sheathed weapon, or sheathe one drawn weapon, or open one door, or other similar actions. It is not used up if you simply drop a held item.

The three actions above can only be used on your turn. Your \textbf{reaction}, however, can be used on anyone's turn. Like bonus actions, reactions can only be used to do specific things. Most characters only need to care about \textit{opportunity attacks}, which can be made when a creature leaves your melee reach.

\subsubsection{Movement}

In addition to the four actions, you can \textbf{move} up to your speed, on your turn. For example, if your speed is 30ft, you can move up to a total of 30ft during your turn, distributed as you wish between your actions.















\end{document}