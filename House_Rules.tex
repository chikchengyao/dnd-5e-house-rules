% bg = [ full | none | print ]
\documentclass[letterpaper,twocolumn,openany,nodeprecatedcode,bg=print]{dndbook}

% Use babel or polyglossia to automatically redefine macros for terms
% Armor Class, Level, etc...
% Default output is in English; captions are located in lib/dndstrings.sty.
% If no captions exist for a language, English will be used.
%1. To load a language with babel:
%	\usepackage[<lang>]{babel}
%2. To load a language with polyglossia:
%	\usepackage{polyglossia}
%	\setdefaultlanguage{<lang>}
\usepackage[english]{babel}
%\usepackage[italian]{babel}
% For further options (multilanguage documents, hypenations, language environments...)
% please refer to babel/polyglossia's documentation.

\usepackage{appendix}
\usepackage[singlelinecheck=false]{caption}
\usepackage[colorlinks = true,
            linkcolor = blue,
            urlcolor  = blue,
            citecolor = blue,
            anchorcolor = blue]{hyperref}
\usepackage[utf8]{inputenc}
\usepackage{lipsum}
\usepackage{listings}
\usepackage{shortvrb}
\usepackage{stfloats}
\usepackage{subfiles}

\MakeShortVerb{|}

\newcommand{\pg}[1]{page \pageref{#1}}
\newcommand{\see}[1]{(see \pg{#1})}

\lstset{%
  basicstyle=\ttfamily,
  language=[LaTeX]{TeX},
  breaklines=true,
}

\setcounter{tocdepth}{2}

\raggedbottom

\begin{document}

%\tableofcontents




\chapter{Core Houserules}

\noindent [Version 8.0.0]

\section{How To Use}
\textbf{The rules on this page are mandatory.}
Other pages can be safely skipped unless this page directs you to them.

The next few pages contain optional rules. These optional rules are designed such that choosing to use them (or not use them) affects only you, no one else. 
So, you are free to use any combination of optional rules, or none of them.

Some optional rules provide mild benefits, while others can be self-imposed challenges. 
None are overly powerful or detrimental. 

\subsubsection{Sources}
You can use any content from the sourcebooks listed on \pg{sources}. 
Ask before using other content, like Unearthed Arcana or homebrew. 

Purely cosmetic changes are usually fine. Balancing will be done on a case-by-case basis, when it seems necessary. 

\subsubsection{New Players}
A primer to D\&D 5e can be found in Appendix \ref{new-players-guide} (\pg{new-players-guide}).

\newpage
\section{Houserules}

\subsection{Races}
Races with feats, like Variant Human (PH, 31) and Custom Lineage (TCE, 8), are not allowed.

\subsection{Ability Scores}
Use standard \href{https://chicken-dinner.com/5e/5e-point-buy.html}{27-point buy} (PH, 13) to determine your initial ability scores, which should be at most 17 after adding racial bonuses.

At \underline{character level} 3, 7, 11, 15, and 19, all characters receive an ASI of 2 points, which cannot be traded for a feat.

In exchange, you do not receive ASIs from your class at \underline{class level} 4, 8, 12, 16 and 19. Instead, you can receive any major feat \see{major-feats-table} or minor feat \see{minor-feats-table}. Any ASIs within these feats (i.e., within half-feats) are ignored.

ASIs received from classes at other class levels work normally. If traded for feats, those feats are the only feats which can increase ability scores.

\subsection{Difficulty Classes}
To succeed, a roll must strictly exceed, not just equal, its DC. 
This includes attacks against AC. 
Reasons are explained in Appendix \ref{difficulty-classes} (\pg{difficulty-classes}).

\subsection{Falling Unconscious}
If you fall unconscious from damage, you gain one level of exhaustion.

\subsection{Prone}
Standing from prone provokes opportunity attacks, which occur just before standing. 
Taking the Disengage action avoids them.

\subsection{Counterspell}
The spell "Counterspell" no longer exists. 
An optional alternative is available \see{counterspell}.

\subsection{Bonus Action Spells}
The rule on Bonus Action Spells (PH, 202) is simplified to
"you can cast at most one levelled spell per turn".

\subsection{Warlocks}
Warlocks are based on Intelligence, not Charisma. 
Also, the Hexblade's Hex Warrior feature (XGE, 55) is transferred to the Pact of the Blade (PH, 107).










\chapter{Character Options}
\noindent You can opt to use any number of these options for your own character.

\subsection{Races}
You may swap your race's ability score increase for a +2/+1 increase to two different ability scores.

\subsection{Backgrounds}
You can use custom backgrounds. Choose:
\begin{itemize}
\item any two skill proficiencies 
\item a total of two tool proficiencies or languages 
\item one equipment package from any background
\item one background feature from any background
\end{itemize}

\subsection{Minor Feats}
\label{minor-feats}
At character level 1, and at class levels 4, 8, 12, 16 and 19, 
you can receive one minor feat \see{minor-feats-table}.

Minor feats usually provide situational benefits and good roleplaying potential.

\subsection{Multiclassing}
Multiclassing (PH, 163) is allowed. 

When multiclassing into or from a Paladin or a Ranger, 
you may use Strength instead of Dexterity (and vice versa) to fulfil the ability score minimum. 

\subsection{Simple HP Formula}
You can opt to use this formula, which is never lower than the standard formula.
\begin{itemize}
\item Your HP without class levels is 5.
\item For every class level, your HP increases by one hit die roll + CON mod. Reroll 1s. 
\item Instead of rolling, you can take the average, rounded up.
\end{itemize}
\subsection{Intelligence Proficiencies}
For every +1 in your base Intelligence modifier (minimum 0), 
you can opt to gain one language or tool proficiency, and one of the following:
\begin{itemize}
    \item proficiency in one INT, WIS or CHA skill, or
    \item upgrade a proficient INT skill to expertise.
\end{itemize}

\subsection{Tool Proficiencies}
When you take your first level of a non-spellcaster class, and every time you take a major feat from it, 
you can gain a tool proficiency, or upgrade an existing tool proficiency to expertise. 

\subsection{Counterspell}
\label{counterspell}
The spell "Counterspell" no longer exists, per the core houserules.
If you are a sorcerer, warlock, wizard, or otherwise could originally access Counterspell,
you can opt to gain this feature at the level when you would normally access it.

If you have five bard levels, you can also gain this feature.

\subsubsection{Counterspell}
If you perceive a spell being cast, 
you can automatically identify it, 
and can attempt to disrupt that triggering spell before it takes effect, using your knowledge of its school of magic.

A counterspell is a spell from the same school as the triggering spell, which you know or have prepared. 
You can cast your counterspell in a modified form to disrupt the triggering spell, instead of producing the spell's usual effect. 
In this modified form, your counterspell can be cast using a reaction, regardless of its normal casting time, 
and requires its usual components, but does not consume them.

Basic knowledge cannot counteract advanced spellwork. 
Therefore, your counterspell cannot be a cantrip, 
and its base level cannot be lower than the triggering spell's base level.

Make a spellcasting ability check against DC 10 + the triggering spell's slot level. 
You gain a bonus to this check equal to the level of the spell slot you used. 
Note that this check is usually made without proficiency.

If your counterspell is the same spell as the triggering spell, you gain advantage on this check, since less modification is needed. 
If your counterspell is cast at a lower level than the triggering spell, you have disadvantage.

\subsection{Vancian Magic}
Prepared spellcasters can opt to be Vancian spellcasters, 
exchanging some spellcasting flexibility for limited access to their entire spell list.

\subsubsection{Preparing Spells}
Instead of preparing a list of spells, you prepare spells inside each of your spell slots. 
The moment you complete a long rest, or otherwise recover spell slots, you must choose the spells occupying each recovered slot. 
You can use a spell slot only to cast the spell occupying it, unless you take the Reprepare action (below) to change the occupying spell.

\begin{DndComment}{Example}
After completing a long rest, a 3rd-level Cleric who uses Vancian magic might have these spells prepared:

\begin{itemize}
\item \textit{Bless}, \textit{Guiding Bolt}, and two \textit{Cure Wounds}, at 1st level.
\item \textit{Cure Wounds} and \textit{Inflict Wounds}, at 2nd level.
\end{itemize}

\noindent This cleric can cast Bless at 1st level, expending that spell. 
After doing so, her remaining prepared spells are:

\begin{itemize}
\item \textit{Guiding Bolt}, and two \textit{Cure Wounds}, at 1st level.
\item \textit{Cure Wounds} and \textit{Inflict Wounds}, at 2nd level.
\end{itemize}

\noindent Now, she can no longer cast Bless because she no longer has it prepared. 

Because Vancian casters can cast spells only at the level they were prepared, 
she can cast \textit{Guiding Bolt} only at 1st level, and \textit{Inflict Wounds} only at 2nd level.
\end{DndComment}

\subsubsection{Repreparing Spells}
You have a number of prep points equal to your spellcaster level, recovering expended uses on a long rest.
These prep points represent the total number of spell levels which you can reprepare.

To reprepare an old spell to become a new spell, first choose the target spell slot containing the old spell, and choose the new spell. You cannot change these choices once repreparation starts.
You then need to take the Reprepare bonus action on a number of consecutive turns equal to the level of the spell slot, spending one prep point each time.
You must concentrate until the start of your first turn after the final Reprepare bonus action, at which point your repreparation completes.
If you lose concentration, the repreparation fails, wasting any expended prep points.

If a feature says you have a spell "always prepared", repreparing any other spell to become that spell does not cost you any prep points.

%\begin{DndTable}[header=Vancian Spell Slots]{r | ccccccccc}
%\textbf{Level} & \textbf{1st} & \textbf{2nd} & \textbf{3rd} & \textbf{4th} & \textbf{5th} & \textbf{6th} & \textbf{7th} & \textbf{8th} & \textbf{9th} \\
%\hline
%1st & 4 & - & - & - & - & - & - & - & - \\
%2nd & 5 & - & - & - & - & - & - & - & - \\
%3rd & 5 & 3 & - & - & - & - & - & - & - \\
%4th & 6 & 4 & - & - & - & - & - & - & - \\
%5th & 6 & 4 & 3 & - & - & - & - & - & - \\
%6th & 6 & 4 & 4 & - & - & - & - & - & - \\
%7th & 6 & 4 & 4 & 2 & - & - & - & - & - \\
%8th & 6 & 5 & 4 & 3 & - & - & - & - & - \\
%9th & 6 & 5 & 4 & 3 & 2 & - & - & - & - \\
%10th & 6 & 5 & 5 & 3 & 3 & - & - & - & - \\
%11th & 6 & 5 & 5 & 3 & 3 & 1 & - & - & - \\
%12th & 6 & 6 & 5 & 4 & 3 & 1 & - & - & - \\
%13th & 6 & 6 & 5 & 4 & 3 & 1 & 1 & - & - \\
%14th & 6 & 6 & 5 & 4 & 3 & 2 & 1 & - & - \\
%15th & 6 & 6 & 5 & 4 & 3 & 2 & 1 & 1 & - \\
%16th & 6 & 6 & 5 & 4 & 3 & 2 & 2 & 1 & - \\
%17th & 6 & 6 & 5 & 4 & 3 & 2 & 2 & 1 & 1 \\
%18th & 6 & 6 & 5 & 4 & 4 & 2 & 2 & 1 & 1 \\
%19th & 6 & 6 & 5 & 4 & 4 & 3 & 2 & 1 & 1 \\
%20th & 6 & 6 & 5 & 4 & 4 & 3 & 2 & 2 & 1 \\
%\end{DndTable}







\chapter{Gameplay Options}
\noindent You can opt to use any number of these to modify your own gameplay.

\subsection{Combat Options}
Besides the standard options like Dash and Shove, 
you can use Disarm, Shove Aside, Climb Onto, Overrun and Tumble (DMG, 271-272). 

For your convenience, combat options are summarised in the Appendix \see{combat-options}.

\subsection{Critical Hits}
\label{critical-hit}
Whenever you roll damage for a critical hit, you can choose to:
\begin{itemize}
\item maximise the dice (no uncertainty)
\item roll the dice twice (default)
\item roll once and double (high risk, high reward)
\end{itemize}

\subsection{Action: Gain a Bonus Action}
\label{gameplay-bonus-action}
You can exchange an action for a bonus action on your turn, 
but you cannot use bonus actions from the same feature more than once in a turn.

\begin{DndComment}{Example}
You can use your action to gain a second bonus action, 
then cast Healing Word with one bonus action 
and use your other bonus action to attack with a Spiritual Weapon. 

However, you may not attack twice with a Spiritual Weapon because this uses two bonus actions from the same spell, 
even if you attack with two separate castings of the spell, 
or if one bonus action was used to attack with an existing Spiritual Weapon while the other was used to cast a new Spiritual Weapon.
\end{DndComment}

\subsection{Hit Dice}
This option gives you more precise control over healing from rests.

You gain two hit dice per class level, instead of one.

However, long rests no longer automatically restore hit points. 
As a long rest ends, you can spend hit dice like in a short rest. 
Then, recover half your expended hit dice, rounded up.

\subsection{Diagonal Grid Movement}
You can choose to move more realistically on a grid. 
Every second diagonal on a turn costs double (DMG, 252). 
Additionally, you have width while moving diagonally. 

\begin{DndComment}{Example}
Because you have width, if you step northeast you will also pass through the north and east grid squares. 
Therefore, you can't step diagonally if either of those squares are blocked, and you provoke opportunity attacks from both squares.
\end{DndComment}

\subsection{Proficiency Dice}
You can opt to use proficiency dice instead of a flat proficiency bonus (DMG, 263).

\subsection{Encumbrance}
You can opt to track encumbrance as a self-imposed challenge. 
You can use any of the three suggested modes below.

\begin{DndTable}[header=Encumbrance (Normal)]{rX}
\textbf{Carried weight} & \textbf{Effect} \\
$\leq$ 15 $\times$ Strength & No penalty \\
$>$ 15 $\times$ Strength & Physically impossible \\
\end{DndTable}

\begin{DndTable}[header=Encumbrance (Hard)]{rX}
\textbf{Carried weight} & \textbf{Effect} \\
$\leq$ 5 $\times$ Strength & No penalty \\
$>$ 5 $\times$ Strength & Cannot travel at a fast pace \\
$>$ 10 $\times$ Strength & All speeds -10ft. \\
$>$ 15 $\times$ Strength & Physically impossible \\
\end{DndTable}

\begin{DndTable}[header=Encumbrance (Impossible)]{rX}
\textbf{Carried weight} & \textbf{Effect} \\
$\leq$ 5 $\times$ Strength & No penalty \\
$>$ 5 $\times$ Strength & All speeds -10ft. \\
$>$ 10 $\times$ Strength & All speeds -20ft., disadvantage on physical checks, attacks and saves. \\
$>$ 15 $\times$ Strength & Physically impossible \\
\end{DndTable}













%\chapter{Clarifications}
%
%\noindent These clarifications must be adhered to, but are rare enough in practice that \textit{reading} them is not mandatory.
%
%\subsection{Feat Preclusions}
%You cannot be a Fighter (Battlemaster) and have the Martial Adept feat.
%
%You cannot have levels in the same class in which you have the Magic Initiate feat.
%
%If such a conflict arises, you can change the conflicting feat.
%
%\subsection{Spellcasting Feats}
%
%\subsubsection{Minimum Level}
%Before you can cast a spell granted by a feat, 
%you must first reach the level at which a full caster can cast that level of spell.
%
%\subsubsection{Learning Spells}
%Ignore any wording in feats that says you "learn" a spell.
%
%\begin{DndComment}{Example}
%If you take the Magic Initiate feat, 
%you can cast the chosen 1st level spell once per long rest, 
%but you do not learn that spell, and therefore cannot cast it using spell slots.
%\end{DndComment}










\appendix
\onecolumn

\chapter{Feats}

\section{Major Feats}
\label{major-feats-table}

Every existing feat is classified as either a major or a minor feat. 

\underline{Underlined text} intentionally overrules source material. 
Remember also that major feats received at class level 4, 8, 12, 16 and 19 do not increase your ability scores. 

You cannot benefit from a class initiate feat if you have levels in the corresponding class (or subclass). If you take a class initiate feat, and later gain a level in that class (or subclass), you can swap the feat for another major feat.

\begin{DndTable}[header=Major Feats]{llll}
    \textbf{Feat} & \textbf{Prerequisite} & \textbf{Source} & \textbf{Summary} \\
    Aberrant Dragonmark & No dragonmark & ERLW, 52 & Learn sorcerer cantrip and L1 spell (short rest). Has side effects. \\
    Crossbow Expert & - & PH, 165 & Ignore 'loading', shoot within 5ft, bonus action shoot. \\
    Crusher & - & TCE, 79 & Bludgeoning damage can shove 5ft, crits grant 1 round advantage. \\
    Defensive Duelist & DEX 13 & PH, 165 & If wielding finesse weapon, use reaction to increase AC. \\
    Dual Wielder & - & PH, 165 & +1 AC, use non-light weapons, draw both weapons at once. \\
    Fey Touched & - & TCE, 79 & Learn Misty Step, and one L1 divination/enchantment spell. \\
    Great Weapon Master & - & PH, 167 & If crit or KO, bonus action attack. -5/+10 to attacks. \\
    Gunner & \underline{ask DM} & TCE, 80 & Learn firearms, ignore 'loading', shoot within 5ft. \\
    Heavily Armored & Medium armor & PH, 167 & Learn heavy armor \underline{and shields}. \\
    Heavy Armor Master & Heavy armor & PH, 167 & Physical damage reduced by 3, \underline{even from magical weapons}. \\
    Lightly Armored & - & PH, 167 & Learn light armor. \\
    Lucky & \underline{ask DM} & PH, 167 & This feat is not recommended. \\
    Medium Armor Master & Medium armor & PH, 168 & Armor doesn't impede stealth, add up to 3 AC from DEX. \\
    Mobile & - & PH, 168 & Speed +10ft, Dash to ignore terrain, attack targets can't OA you. \\
    Moderately Armored & Light armor & PH, 168 & Learn medium armor and shields. \\
    Piercer & - & TCE, 80 & \underline{Piercing damage crits on 19}, and crits roll one more damage die. \\
    Polearm Master & - & PH, 168 & Bonus action d4 attack. Entering your reach triggers OA. \\
    Resilient & - & PH, 168 & Gain proficiency in \underline{DEX/CON/WIS save and STR/INT/CHA save}. \\
    Revenant Blade & Elf & ERLW, 22 & +1 AC while using double-bladed scimitar. Counts as finesse. \\
    Savage Attacker & \underline{ask DM} & PH, 169 & This feat is not recommended. \\
    Sentinel & - & PH, 169 & Your OAs cripple, ignore Disengage. OA if enemy attacks friend. \\
    Sharpshooter & - & PH, 170 & Increased range, ignore cover, -5/+10 to attacks. \\
    Slasher & - & TCE, 81 & Slashing damage -10ft speed, crits grant 1 round disadvantage. \\
    Tough & - & PH, 170 & Extra 2 max HP per level, \underline{and hit dice heal an extra 2 HP each}. \\
    War Caster & - & PH, 170 & Advantage on concentration. Somatic with hands full. OA spells. \\
\end{DndTable}    
    
\begin{DndTable}[header=Major Feats (Class Initiate Feats)]{llll}
    \textbf{Feat} & \textbf{Prerequisite} & \textbf{Source} & \textbf{Summary} \\
    \underline{Artificer Initiate} & \underline{INT 11} & \underline{\pg{artificer-initiate}} & Learn one cantrip and \underline{two} tools. \underline{Prepare} one L1 spell. \\
    Infuser Initiate (NEW) & INT 11 & \pg{infuser-initiate} & Learn two artificer infusions, infuse one. \\
    Barbarian Initiate (NEW) & STR 11 & \pg{barbarian-initiate} & Rage once per long rest. \\
    Bard Initiate (NEW) & CHA 11 & \pg{bard-initiate} & Learn two cantrips, one L1 spell, and gain two d4 Bardic Inspiration. \\
    Cleric Initiate (NEW) & WIS 11 & \pg{cleric-initiate} & Learn two cantrips, one L1 spell, and gain mini Divine Intervention. \\
    Divine Initiate (NEW) & WIS/CHA 11 & \pg{divine-initiate} & Channel Divinity once per long rest. \\
    Druid Initiate (NEW) & WIS 11 & \pg{druid-initiate} & Learn two cantrips, one L1 spell, and gain Wild Shape. \\
    Fighting Initiate & \underline{STR/DEX 11} & TCE, 80 & Learn one Fighting Style. \\
    Martial Adept & \underline{STR/DEX 11} & PH, 168 & Learn two Battlemaster manoeuvres, gain \underline{two superiority d8}. \\
    Monk Initiate (NEW) & DEX\&WIS 11 & \pg{monk-initiate} & 2 Ki, 1d4 punch using STR/DEX, Patient Defence, Step of the Wind. \\
    Paladin Initiate (NEW) & (STR/DEX)\&CHA 11 & \pg{paladin-initiate} & 5 HP Lay on Hands, and smite once per long rest. \\
    Ranger Initiate (NEW) & (STR/DEX)\&WIS 11 & \pg{ranger-initiate} & Prepare one L1 spell, learn ranger skills, understand plants/beasts. \\
    Rogue Initiate (NEW) & DEX 11 & \pg{rogue-initiate} & Bonus action Hide. One skill, expertise, thieves' cant, thieves' tools. \\
    Sorcerer Initiate (NEW) & CHA 11 & \pg{sorcerer-initiate} & Learn three cantrips. Learn one L1 spell, cast once as Subtle Spell. \\
    Metamagic Adept & \underline{CHA 11} & TCE, 80 & Learn two metamagics, gain 2 sorcery points. \\
    Warlock Initiate (NEW) & INT 11 & \pg{warlock-initiate} & Learn two cantrips, and one L1 spell regained on a short rest. \\
    Eldritch Adept & \underline{INT 11} & TCE, 79 & Learn one warlock invocation. \\
    Wizard Initiate (NEW) & INT 11 & \pg{wizard-initiate} & Learn two cantrips, one L1 spell, and ritual cast Find Familiar. \\
    
\end{DndTable}

\newpage
\section{Minor Feats}
\label{minor-feats-table}

Every existing feat is classified as either a major or a minor feat. 

\underline{Underlined text} intentionally overrules source material. 
Remember also that minor feats received at character level 1 and at class level 4, 8, 12, 16 and 19 do not increase your ability scores. 

\begin{DndTable}[header=Minor Feats]{llll}
    \textbf{Feat} & \textbf{Prerequisite} & \textbf{Source} & \textbf{Summary} \\
    Actor & - & PH, 165 & Disguise as others, mimic speech and sounds. \\
    Alert & - & PH, 165 & +5 initiative, immune to surprise, prevent unseen attacks. \\
    Athlete & - & PH, 165 & Easily climb, jump, and stand from prone. \\
    Bountiful Luck & Halfling & XGE, 73 & Use reaction to extend Lucky to allies. \\
    Charger & - & PH, 165 & If you Dash, also make a buffed attack or shove. \\
    Chef & - & TCE, 79 & Learn cook's tools, and cook healing food. \\
    Dragon Fear & Dragonborn & XGE, 74 & 30ft radius frighten using breath weapon. \\
    Dragon Hide & Dragonborn & XGE, 74 & Gain natural armour and d4 claws. \\
    Drow High Magic & Elf (Drow) & XGE, 74 & Learn Detect Magic (at will), Levitate, and Dispel Magic. \\
    Dungeon Delver & - & PH, 166 & Find secret doors, and effectively avoid or resist traps. \\
    Durable & - & PH, 166 & Hit dice heal more reliably. \\
    Dwarven Fortitude & Dwarf & XGE, 74 & Heal with hit dice while Dodging. \\
    Elemental Adept & Spells & PH, 166 & Spells of chosen element are buffed and ignore resistance. \\
    Elven Accuracy & Elf / Half-elf & XGE, 74 & Roll 3 d20s for advantage on DEX/INT/WIS/CHA attacks. \\
    Fade Away & Gnome & XGE, 74 & One round invisibility when damaged, once per short rest. \\
    Fey Teleportation & Elf (High) & XGE, 74 & Learn Sylvan and Misty Step. \\
    Flames of Phlegethos & Tiefling & XGE, 74 & Buff fire spell damage, wreathe self in flame. \\
    Grappler & STR 13 & PH, 167 & Advantage to hit grapplee, learn to restrain grapplee. \\
    Healer & - & PH, 167 & Heal everyone once per short rest, heal when stabilising. \\
    Infernal Constitution & Tiefling & XGE, 75 & Resist cold, poison, and 'poisoned'. \\
    Inspiring Leader & CHA 13 & PH, 167 & Take 10 min break to grant temp HP, once per short rest. \\
    Keen Mind & - & PH, 167 & Find north, always know the time, photographic memory. \\
    Linguist & - & PH, 167 & Learn three languages, and craft written ciphers. \\
    Mage Slayer & - & PH, 168 & Harass casters effectively in melee. \\
    Mounted Combatant & - & PH, 168 & Buff mounted attacks and protect your mount. \\
    Observant & - & PH, 168 & Read lips, +5 passive perception and investigation. \\
    Orcish Fury & Half-Orc & XGE, 75 & Extra damage once per short rest, and attack when KO'ed. \\
    Poisoner & - & TCE, 80 & Learn to craft, quickly apply, and ignore resistance to poison. \\
    Prodigy & Human / Half-Elf / Half-Orc & XGE, 75 & Learn one language, skill, tool, expertise. \\
    Ritual Caster & \underline{Casting ability 11} & PH, 169 & Learn two L1 ritual spells from one caster class. \\
    Second Chance & Halfling & XGE, 75 & Force one attack to reroll, once per combat. \\
    Shadow Touched & - & TCE, 80 & Learn Invisibility, and one L1 illusion/necromancy spell. \\
    Shield Master & - & PH, 170 & Better DEX saves, bonus action shove with shield. \\
    Skill Expert & - & TCE, 80 & Learn one skill and one expertise. \\
    Skilled & - & PH, 170 & Learn three skills or tools. \\
    Skulker & DEX 13+ & PH, 170 & Hide and see in dim light, missed attacks don't break stealth. \\
    Spell Sniper & Spells, \underline{Casting ability 11} & PH, 170 & Learn one attack cantrip, attack range doubled, ignore cover. \\
    Squat Nimbleness & Dwarf or Small race & XGE, 75 & +5ft speed, learn Ath/Acr, easily escape grapples. \\
    Stunning Irruption (NEW) & - & \pg{stunning-irruption} & Initiate combat by bursting through door and stunning. \\
    Svirfneblin Magic & Gnome (Deep) & MTF, 114 & Learn Nondetection (at will), Blind/Deaf, Blur, Disguise Self. \\
    Tavern Brawler & - & PH, 170 & Improvise weapons, d4 punch, bonus action grapple after punch. \\
    Telekinetic & - & TCE, 81 & Invisible mage hand, and telekinetic shove. \\
    Telepathic & - & TCE, 81 & One way telepathic speech. Learn Detect Thoughts. \\
    Tis But A Scratch (NEW) & - & \pg{tis-but-a-scratch} & When hit by crit, attempt to frighten opponent. \\ 
    Weapon Master & - & PH, 170 & Learn four weapons. \\
    Well Prepared (NEW) & - & \pg{well-prepared} & Retroactively declare that you bought an item. \\
    Wood Elf Magic & Elf (Wood) & XGE, 75 & Learn one druid cantrip, Longstrider, Pass Without Trace. \\
\end{DndTable}



\twocolumn

\newpage

\section{New Feats}

\DndFeatHeader{Artificer Initiate}[Prerequisite: Intelligence 11 or higher]
\label{artificer-initiate}
You gain proficiency with two types of tools, which can be thieves' tools or artisan's tools.

You learn one artificer cantrip. 
At the end of each long rest, choose one 1st-level spell from the artificer spell list to prepare (replacing your previously prepared spell, if any), which you can cast once without using a spell slot. 
You can also cast this spell using spell slots. 
Intelligence is your spellcasting ability for these spells.
\DndFeatHeader{Barbarian Initiate}[Prerequisite: Strength 11 or higher]
\label{barbarian-initiate}
You can rage (PH, 48) once per long rest.

\DndFeatHeader{Bard Initiate}[Prerequisite: Charisma 11 or higher]
\label{bard-initiate}
You gain the Magic Initiate feat (PH, 168) for the Bard class.

Also, you gain two d4 Bardic Inspiration dice (PH, 53), regaining them on a long rest.

\DndFeatHeader{Cleric Initiate}[Prerequisite: Wisdom 11 or higher]
\label{cleric-initiate}
You gain the Magic Initiate feat (PH, 168) for the Cleric class.

Also, you can ask your deity for help, using a bonus action. 
Point to a target you can see within 60 feet, and say either "weal" or "woe". 
Roll a d20, succeeding if you roll a number equal to or lower than your level.
The DM chooses the nature of the intervention.
The effect of a cleric spell or domain spell up to half your proficiency bonus (rounded up) would be appropriate. 
You can succeed only once per long rest.

\DndFeatHeader{Divine Initiate}[Prerequisite: Wisdom or Charisma 11 or higher]
\label{divine-initiate}
Choose one kind of Channel Divinity. 
If you meet the Wisdom prerequisite, you can choose from any cleric domain. 
If you meet the Charisma prerequisite, you can choose from any paladin oath. 

You can use the chosen Channel Divinity once per long rest.

\DndFeatHeader{Druid Initiate}[Prerequisite: Wisdom 11 or higher]
\label{druid-initiate}
You gain the Magic Initiate feat (PH, 168) for the Druid class.

Also, you can Wild Shape (PH, 66) once per short rest, as if you were a 2nd-level druid.

\DndFeatHeader{Infuser Initiate}[Prerequisite: Intelligence 11 or higher]
\label{infuser-initiate}
You have learned to infuse items like an artificer (TCE, 12). 
You learn two artificer infusions and can infuse one item.

\DndFeatHeader{Monk Initiate}[Prerequisite: Dexterity and Wisdom 11 or higher]
\label{monk-initiate}
Your unarmed strikes can use 1d4 damage dice, and can use Dexterity instead of Strength. You gain 2 ki points, which you can use for Patient Defense or Step of the Wind (PH, 78). You regain ki points on a short rest.

\DndFeatHeader{Paladin Initiate}[Prerequisite: Strength or Dexterity 11 or higher, and Charisma 11 or higher]
\label{paladin-initiate}
You gain Lay On Hands (PH, 84) with a pool of 5 hit points. Once per long rest, you can deal an extra 2d8 radiant damage when you hit with a melee weapon attack.

\DndFeatHeader{Ranger Initiate}[Prerequisite: Strength or Dexterity 11 or higher, and Wisdom 11 or higher]
\label{ranger-initiate}
At the end of each long rest, choose one 1st-level spell from the ranger spell list to prepare (replacing your previously prepared spell, if any), which you can cast once without using a spell slot. 

Additionally, choose two skills from this list: Animal Handling, Nature, Stealth, and Survival. You gain proficiency with these skills. If you are already proficient with all four, you can choose any skill.

You also have advantage on Wisdom (Survival) checks to find, examine and follow tracks.

Finally, you can understand simple ideas which plants and beasts communicate to you. This does not grant you any ability to communicate to them.

\DndFeatHeader{Rogue Initiate}[Prerequisite: Dexterity 11 or higher]
\label{rogue-initiate}
You can Hide as a bonus action. You also learn thieves' cant (PH, 96), and gain proficiency in thieves' tools and any one skill from the rogue's skill list (PH, 95). Finally, choose one proficiency (with thieves' tools or in a rogue skill) to upgrade to expertise.

\DndFeatHeader{Sorcerer Initiate}[Prerequisite: Charisma 11 or higher]
\label{sorcerer-initiate}
You gain the Magic Initiate feat (PH, 168) for the Sorcerer class, except that you learn three sorcerer cantrips instead of two. Additionally, when you cast the 1st-level spell from this feat without using a spell slot, you can choose to cast it without any somatic or verbal components.

\DndFeatHeader{Stunning Irruption}
\label{stunning-irruption}
You can begin a combat by slamming open a door, or making a similar entrance. All surprised creatures within 30 feet of the door must succeed on a Constitution save or be stunned while they are surprised. The DC is 8 + your STR mod + your proficiency bonus.

\DndFeatHeader{Tis But A Scratch}
\label{tis-but-a-scratch}
Whenever an opponent critically hits you, you can use your reaction to make a Constitution (Deception or Intimidation) check, contested by the opponent's Wisdom (Medicine) check. If you win, your opponent is frightened until the end of its next turn.

\DndFeatHeader{Warlock Initiate}[Prerequisite: Intelligence 11 or higher]
\label{warlock-initiate}
You gain the Magic Initiate feat (PH, 168) for the Warlock class, except that you can cast the 1st-level spell from this feat once per short rest without using a spell slot, instead of once per long rest.

\DndFeatHeader{Well Prepared}
\label{well-prepared}
When confronted with a situation that calls for a mundane item, you may make an Intelligence or Wisdom check with a DC of 5 plus the item's gp value. On a success, you "happen" to have the item, bought some time ago. Deduct the price of the item from your money. 

Once you succeed, you cannot succeed again until you finish a long rest.

\DndFeatHeader{Wizard Initiate}[Prerequisite: Intelligence 11 or higher]
\label{wizard-initiate}
You gain the Magic Initiate feat (PH, 168), for the Wizard class. Additionally, you can cast Find Familiar as a ritual.



\chapter{Tables}

\section{Sources}
\label{sources}
\begin{DndTable}{rl}
%\begin{DndTable}[header=\href{https://thetrove.is/Books/Dungeons\%20\%26\%20Dragons\%20\%5Bmulti\%5D/5th\%20Edition\%20\%285e\%29/
%Core/}{Official Sourcebooks}]{rl}
\textbf{Abbv.} & \textbf{Name} \\
DMG & Dungeon Master's Guide \\
ERLW & Eberron - Rising from the Last War \\
EE & Elemental Evil Player's Companion \\
EGW & Explorer's Guide to Wildemount \\
GGR & Guildmaster's Guide to Ravnica \\
MM & Monster Manual \\
MTF & Mordenkainen's Tome of Foes \\
PH & Player's Handbook \\
SCAG & Sword Coast Adventurer's Guide \\
TCE & Tasha's Cauldron of Everything \\
VGM & Volo's Guide to Monsters \\
XGE & Xanathar's Guide to Everything \\
\end{DndTable}


\section{Combat Options}
\label{combat-options}
\begin{DndTable}[header=Standard]{llX}
\textbf{Cost} & \textbf{Option} & \textbf{Brief description} \\
Action & Attack & You can make 1 weapon attack. \\
Action & Cast Spell & Refer to spell description. \\
Action & Dash & Gain extra movement on this turn equal to your speed. \\
Action & Disengage & Avoid triggering opportunity attacks. \\
Action & Dodge & Until next turn, disadvantage to be hit, and advantage on Dex saves. \\
Action & Help & Give someone else advantage on their next check/attack. \\
Action & Hide & Go from "not visible" to "not there". \\
Action & Ready & Hold any other action until triggered. \\
Bonus & Improvise & Do something else (check with DM). \\
Bonus & Search & Find a hidden object or creature. \\
1 attack & Grapple & Grab opponent (+1 size). (contest Ath vs. Ath/Acr) \\
1 attack & Shove & Push 5ft. away or knock prone (+1 size). (contest Ath vs. Ath/Acr)\\
\end{DndTable}

\begin{DndTable}[header=Advanced]{llX}
\textbf{Cost} & \textbf{Option} & \textbf{Brief description} \\
Action & Gain B.A. & Gain another bonus action. \\
1 attack & Disarm & Attempt to knock weapon or item from opponent's hands. (Attack vs. Ath/Acr) \\
1 attack & Shove Aside & Attempt to push opponent 5ft. to the side (+1 size). (Ath[disadv] vs. Ath/Acr) \\
Half speed & Climb Onto & Attempt to climb onto creature (+2 size), gaining advantage on attacks against it. You may be thrown off. (Ath vs. Acr) \\
5ft. & Overrun & Attempt to charge through opponent's space. (Ath vs. Ath/Acr) \\
5ft. & Tumble & Attempt to slip through opponent's space. (Acr vs. Ath/Acr) \\
\end{DndTable}

\section{Allowed Races}
\label{official-races}
\begin{DndTable}{lXl}
    \textbf{Race} & \textbf{Subrace(s)} & \textbf{Source} \\
    Aarakocra & - & EEPC, 5 \\
    Aasimar & Fallen, Protector, Scourge & VGM, 104 \\
    Bugbear & - & ERLW, 25 \\
    Centaur & - & GGR, 15 \\
    Changeling & - & ERLW, 18 \\
    Dragonborn & - & PH, 34 \\
    Dwarf & Hill, Mountain & PH, 20 \\
      & Duergar & SCAG, 104 \\
      & Mark of Warding & ERLW, 51 \\
    Elf & Dark/Drow, High, Wood & PH, 23 \\
      & Eladrin, Sea, Shadar-kai & MTF, 62-63 \\
      & Mark of Shadow & ERLW, 49 \\
    Firbolg & - & VGM, 107 \\
    Genasi & Air, Earth, Fire, Water & EEPC, 9 \\
    Gith & Githyanki, Githzerai & MTF, 96 \\
    Gnome & Forest, Rock & PH, 36 \\
      & Deep & SCAG, 115 \\
      & Mark of Scribing & ERLW, 47 \\
    Goblin & - & ERLW, 26 \\
    Goliath & - & VGM, 109 \\
    Half-Elf & Standard & PH, 39 \\
      & Specific elven descent & SCAG, 116 \\
      & Marks of Detection, Storm & ERLW, 40,50 \\
    Half-Orc & - & PH, 41 \\
      & Mark of Finding & ERLW, 41 \\
    Halfling & Lightfoot, Stout & PH, 28 \\
      & Ghostwise & SCAG, 110 \\
      & Marks of Healing, Hospitality & ERLW, 43-44 \\
    Hobgoblin & - & ERLW, 26 \\
    Human & - & PH, 31 \\
      & Marks of Finding, Handling, Making, Passage, Sentinel & ERLW, 41-48 \\
    Kalashtar & - & ERLW, 30 \\
    Kenku & - & VGM, 111 \\
    Kobold & - & VGM, 119 \\
    Lizardfolk & - & VGM, 113 \\
    Loxodon & - & GGR, 18 \\
    Minotaur & - & GGR, 19 \\
    Orc & - & ERLW, 32 \\
    Shifter & Beasthide, Longtooth, Swiftstride, Wildhunt & ERLW, 33-34 \\
    Simic Hybrid & - & GGR, 20 \\
    Tabaxi & - & VGM, 115 \\
    Tiefling & Standard (aka Asmodeus) & PH, 43 \\
      & Devil's Tongue, Feral, Hellfire & SCAG, 118 \\
      & Winged & SCAG, 118 \\
      & Baalzebul, Dispater, Fierna, Glasya, Levistus, Mammon, Mephistopheles, Zariel & MTF, 21-23 \\
    Triton & - & VGM, 117 \\
    Vedalken & - & GGR, 21 \\
    Warforged & - & ERLW, 36 \\
    Yuan-ti & - & VGM, 120 \\
\end{DndTable}










\chapter{Difficulty Classes}
\label{difficulty-classes}

Under this new rule, a roll must strictly exceed its DC to succeed. 
This includes attack rolls against AC. 
This change aims to make DCs more intuitive and reversible.

\subsection{Intuitive DCs}

Under the standard rules, an average human with 10 in every ability would have the following success rates on checks:

\begin{itemize}
\item 80\% on a DC5 check,
\item 55\% on a DC10 check,
\item 30\% on a DC15 check.
\end{itemize}

\noindent Under the core houserules, the success rates change to the following, which is easier to intuit:

\begin{itemize}
\item 75\% on a DC5 check,
\item 50\% on a DC10 check,
\item 25\% on a DC15 check.
\end{itemize}

\noindent Just like how that average human's +0 modifier has a 50\% chance against DC10, a character with a +5 modifier would have a 50\% chance against DC15, and so on.

Additionally, a +7 roll against DC15 has a 60\% chance of succeeding, which is easily intuited as two 5\% steps above 50\%. 
Contrast this to the 65\% chance under the normal rules.

\subsection{Reversible DCs}
Under the core houserules, a vampire with +7 to attack against a rogue's AC15 has a 60\% chance to hit the rogue. 
From the rogue's point of view, he has 40\% chance to defend. 

But suppose that instead of the vampire rolling to attack, the rogue could roll to defend. We could convert the rogue's AC15 into a +5 modifier to defend, while the vampire's +7 to hit is equivalent to DC17. Such a +5 'defense roll' by the rogue against the vampire's DC17 has a 40\% chance to succeed, exactly the same chance that the vampire misses a +7 attack roll against the rogue's AC15.

This reversal doesn't work under the original DC rules. But under the new rule, we can easily flip any attack roll by the DM into a defence roll by the player, and any saving throw could be converted into a 'forcing throw'.

This may potentially be used in the future to let players make more dice rolls, increasing player agency and reducing DM workload.

\section{Drawbacks}
This houserule shifts combat balance somewhat, weakening attacks by 5\% and strengthening abilities that force a saving throw by 5\%. 





\chapter{New Player's Guide}
\label{new-players-guide}

% Basics
\newpage
\section{Characters}

All characters have six \textbf{abilities}: Strength, Dexterity, Constitution, Intelligence, Wisdom, and Charisma. 
They represent how good characters are at different types of tasks. 
Characters also receive \textbf{features} from:

\begin{itemize}
\item their \textbf{race}, like "human" or "elf",
\item their \textbf{class}, like "paladin" or "wizard",
\item their \textbf{background}, like "entertainer" or "spy".
\end{itemize}

\subsection{Abilities}
Your character has a score for each ability. 
An ability score of 10 is the human average.

Ability scores translate to a \textbf{modifier}. 
These modifiers are much more important than their underlying scores, because modifiers are added directly to dice rolls. 
An ability score of 10 corresponds to a modifier of 0. 
Every 2 extra points in an ability score grants a +1 to your modifier. 

\begin{DndTable}[]{ll}
    \textbf{Ability score} & \textbf{Modifier} \\
    8-9 & –1 \\
    10-11 & ±0 \\
    12-13 & +1 \\
    14-15 & +2 \\
    ... & ... \\
\end{DndTable}

\begin{DndComment}{Intelligence and Wisdom}

\noindent Intelligence (book-smarts) and Wisdom (street-smarts) are sometimes confused. 
They can often achieve the same goal, but these examples illustrate their differences:
\begin{itemize}
\item Intelligence helps you deduce the optimal path through a maze, while Wisdom helps you spot hidden shortcuts.
\item Intelligence helps you recall a creature's natural habitat, while Wisdom helps you track the creature through it.
\item Intelligence helps you search a crime scene for clues, while Wisdom helps you realise they were purposely left.
\end{itemize}
\end{DndComment}

\subsection{Race, Class, Background}
Your race, class and background give you \textbf{features} which differentiate you from other characters. 

Racial features represent your culture of origin and genes. 
For example, dwarves are short and slow, but have darkvision, can speak Dwarvish, and have learned to wield warhammers. 

Class features represent your actively developed capabilities. 
For example, a Fighter will continually improve their skill at arms, 
while a druid continually gains more powerful spells as she grows more attuned with nature. 

Background features represent your personal history. 
For example, a spy would have learned to pick locks and honed their deceptiveness.


% d20 rolls
\newpage
\section{D20 Rolls}

A d20 is a 20-sided die. 
There are three kinds of d20 rolls: \textbf{checks}, \textbf{attacks}, and \textbf{saves}.

\subsection{Checks}
Ability \textbf{checks}, or just "checks", are attempts to do something difficult, 
like bluff a guard, search for a hidden door, or cheat at a card game. 
Checks are tied to an ability. 
Roll a \textbf{d20}, and add the associated ability modifier. 
For example, if your Intelligence is 16, your modifier is +3. 
Therefore, your Intelligence check will be d20 + 3, which is at least 4 and at most 23.

If your check exceeds the task's Difficulty Class (or "DC"), you succeed.

\begin{DndComment}{Proficiency}

\noindent Special training or experience is represented by \textit{proficiency}. 
For example, a 1st-level character proficient in History could add their proficiency bonus of +2 to any History-related check. 
Your proficiency bonus increases slowly as you level up.

\textbf{Proficiency can only add to d20 rolls, by default}. 
You do not add proficiency to damage rolls, for instance.

\end{DndComment}

\subsection{Attacks}
An \textbf{attack} is an attempt to aim a weapon at a small target, such as a gap in an enemy's armour. 
Roll a \textbf{d20}, and add your Strength modifier for melee weapons or Dexterity modifier for ranged ones. 
If you are proficient with the weapon used, add your proficiency bonus.

If your attack roll exceeds the target's Armour Class (or "AC"), then your attack hits, and you can make the weapon's damage roll (e.g. 1d6+3).

A special feature of attacks is that a natural 1 on the d20 always misses, and a natural 20 is a critical hit \see{critical-hit}, doing extra damage.

\subsection{Saves}
A saving throw, usually called a "\textbf{save}", is an instinctive, involuntary action to save yourself from an effect. 
Saves are tied to an ability. 
For example, you might roll a Constitution save to resist poison, or a Wisdom save to resist mind control. 
Roll a \textbf{d20}, and add the linked ability modifier. 
If you are proficient in the save of the associated ability, add your proficiency bonus.
If your save exceeds the effect's Difficulty Class (or "DC"), you succeed.

Proficiency in saves is much more difficult to obtain than proficiency in checks or attacks. 
Most characters will only ever have the two save proficiencies they started with.





\section{Combat}
Combat occurs in \textbf{rounds} lasting 6 seconds each, so that there are 10 rounds in a minute. 
During a round, every participant gets a \textbf{turn}. 
Turn order is determined at the start of combat, by \textit{rolling for initiative}.

\subsection{Actions}
At the start of each of your turns, you get:

\begin{itemize}
\item one \textbf{action}
\item one \textbf{bonus action}
\item one \textbf{object interaction}
\item one \textbf{reaction}
\end{itemize}

You use these to sort-of "buy" things to do, usually standard actions from the "Combat Options" list \see{combat-options}.

Note that the word "action" can be confusing, because it can refer to both the "currency" and the "thing to do" which you buy. Even when used to mean just "currency", the word "action" can specifically mean the "action", but also sometimes generically refers to any of the four.

Your \textbf{action} is the most powerful and flexible of the four. 
It is usually used to Attack or Cast A Spell, but you can take any actions on the standard list, 
or actions from your class or race.

Your \textbf{bonus action} is essentially a minor action, but can only be used to do certain specific things. 
Barbarians need to use a bonus action to enter a rage, while a Wizard might never use a bonus action. 
Many characters start with nothing they can use their bonus action for.

An \textbf{object interaction} can be used to 
draw one sheathed weapon, 
or sheathe one drawn weapon, 
or open one door, 
or do other similar things. 
It is not used up if you simply drop a held item. But if you want to perform two object interactions on the same turn, you'll need to spend your bonus action or main action to do the second.

The three actions above can only be used on your turn. 
Your \textbf{reaction}, however, must be \textit{triggered}, which usually happens on someone else's turn. 
Like bonus actions, reactions can only be used to do specific things. 
Most characters only need to care about making \textit{opportunity attacks}, which trigger on creatures which move out of your melee range.

\subsubsection{Movement}

In addition to the four actions, you gain \textbf{movement} equal to your \textbf{speed} stat on your turn. 
For example, if your speed is 30ft, you gain 30ft movement to spend during your turn, distributed as you wish between your actions. You can gain additional movement equal to your speed by spending your action to \textit{Dash} (see "Combat Options" on \pg{combat-options}).















\end{document}