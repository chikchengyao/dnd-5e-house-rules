\documentclass[letterpaper,twocolumn,openany,nodeprecatedcode]{dndbook}

% Use babel or polyglossia to automatically redefine macros for terms
% Armor Class, Level, etc...
% Default output is in English; captions are located in lib/dndstrings.sty.
% If no captions exist for a language, English will be used.
%1. To load a language with babel:
%	\usepackage[<lang>]{babel}
%2. To load a language with polyglossia:
%	\usepackage{polyglossia}
%	\setdefaultlanguage{<lang>}
\usepackage[english]{babel}
%\usepackage[italian]{babel}
% For further options (multilanguage documents, hypenations, language environments...)
% please refer to babel/polyglossia's documentation.

\usepackage[utf8]{inputenc}
\usepackage[singlelinecheck=false]{caption}
\usepackage{hyperref}
\usepackage{lipsum}
\usepackage{listings}
\usepackage{shortvrb}
\usepackage{stfloats}
\usepackage{subfiles}

\MakeShortVerb{|}

\newcommand{\pg}[1]{pg.\pageref{#1}}
\newcommand{\see}[1]{(see \pg{#1})}

\lstset{%
  basicstyle=\ttfamily,
  language=[LaTeX]{TeX},
  breaklines=true,
}

\setcounter{tocdepth}{2}

\raggedbottom

\begin{document}

%\tableofcontents




\chapter{Core Rules}

[Version 7.2.0]

\section{How To Use}
The only mandatory rules are these core rules on this page. These are simple changes which fix important flaws in the game.

\subsubsection{Optional Rules}
Everything else in this document is \textit{optional.} Some optional rules provide flexibility or flavour, while others can be self-imposed challenges. None are overly powerful or detrimental. They are designed so that each player has the freedom to choose to use any subset for themselves.

\subsubsection{Sources}
You can use any material from the official sourcebooks. Copies are available in the Appendix \see{sources}. 

Unofficial material, such as Unearthed Arcana or homebrewed classes, is probably fine. But ask first.

\subsubsection{Balancing}
Balancing will be done within a party, on a case-by-case basis, when it seems necessary. Trying to balance all classes/subclasses and races against each other in a vacuum is probably impossible. 

\newpage
\section{Houserules}

\subsection{Base Ability Scores}
Use standard 27-point buy (PH, 13) for your ability scores. 

At 1st level, ability scores are capped at 17, after racial bonuses.

\subsection{Falling Unconscious}
If you fall unconscious from damage, you gain one level of exhaustion.

\subsection{Prone}
Standing from prone provokes opportunity attacks. Taking the Disengage action avoids this.
These opportunity attacks occur while still prone.

\subsection{Counterspell}
The spell "Counterspell" no longer exists. An \textit{optional} alternative is available \see{counterspell}.

\subsection{Warlocks}
Warlocks are based on Intelligence, not Charisma.










\chapter{Character Options}
You can opt to use any number of these for your own character.

\subsection{Simple HP Formula}
You can opt to use this formula, which is never lower than the standard formula.
\begin{itemize}
\item Your HP without class levels is 5.
\item For every class level, your HP increases by one hit die roll + CON mod. Reroll 1s. 
\item Instead of rolling, you can take the average, rounded up.
\end{itemize}

\subsection{Intelligence Proficiencies}
For every +1 in your base Intelligence modifier (minimum 0), you can opt to learn one language and one of the following:
\begin{itemize}
    \item learn one tool, or
    \item learn one Int, Wis or Cha skill, or
    \item double your proficiency in an Int skill.
\end{itemize}

\subsection{Races}
If your race gives a +2 and +1 increase to two fixed ability scores, you may opt to swap the two.

\begin{DndComment}{Example}
\begin{itemize}
\item High Elves and Forest Gnomes can start with +2 Dex and +1 Int, or +1 Dex and +2 Int.
\item Humans and Mountain Dwarves are not in +2/+1 format.
\item Changelings and Warforged do not gain fixed ability scores.
\end{itemize}

\end{DndComment}

\subsection{Multiclassing}
Multiclassing (PH, 163) is allowed. 

When multiclassing into or from a Barbarian, Paladin or Ranger, you may use Strength instead of Dexterity (and vice versa) to fulfil the ability score minimum. 

%\subsection{Vancian Magic}
%Prepared spellcasters can opt to be Vancian spellcasters. During daily preparations, you prepare spells directly into spell slots, which cannot be changed until the next preparation. In return for the reduced flexibility, you gain additional spell slots, described in the table.
%
%If you normally have a spell "always prepared", you can convert any other prepared spell to that spell by concentrating for 1 minute per spell level.
%
%Arcane Recovery and Natural Recovery, instead of recovering expended spell slots, allows you to re-prepare spell slots with a combined level equal to or less than your level. You cannot re-prepare spells of 6th level or higher. 
%
%\begin{DndTable}[header=Vancian Spell Slots]{r | ccccccccc}
%\textbf{Level} & \textbf{1st} & \textbf{2nd} & \textbf{3rd} & \textbf{4th} & \textbf{5th} & \textbf{6th} & \textbf{7th} & \textbf{8th} & \textbf{9th} \\
%\hline
%1st & 4 & - & - & - & - & - & - & - & - \\
%2nd & 5 & - & - & - & - & - & - & - & - \\
%3rd & 5 & 3 & - & - & - & - & - & - & - \\
%4th & 6 & 4 & - & - & - & - & - & - & - \\
%5th & 6 & 4 & 3 & - & - & - & - & - & - \\
%6th & 6 & 4 & 4 & - & - & - & - & - & - \\
%7th & 6 & 4 & 4 & 2 & - & - & - & - & - \\
%8th & 6 & 5 & 4 & 3 & - & - & - & - & - \\
%9th & 6 & 5 & 4 & 3 & 2 & - & - & - & - \\
%10th & 6 & 5 & 5 & 3 & 3 & - & - & - & - \\
%11th & 6 & 5 & 5 & 3 & 3 & 1 & - & - & - \\
%12th & 6 & 6 & 5 & 4 & 3 & 1 & - & - & - \\
%13th & 6 & 6 & 5 & 4 & 3 & 1 & 1 & - & - \\
%14th & 6 & 6 & 5 & 4 & 3 & 2 & 1 & - & - \\
%15th & 6 & 6 & 5 & 4 & 3 & 2 & 1 & 1 & - \\
%16th & 6 & 6 & 5 & 4 & 3 & 2 & 2 & 1 & - \\
%17th & 6 & 6 & 5 & 4 & 3 & 2 & 2 & 1 & 1 \\
%18th & 6 & 6 & 5 & 4 & 4 & 2 & 2 & 1 & 1 \\
%19th & 6 & 6 & 5 & 4 & 4 & 3 & 2 & 1 & 1 \\
%20th & 6 & 6 & 5 & 4 & 4 & 3 & 2 & 2 & 1 \\
%\end{DndTable}
%
\subsection{Backgrounds}
You can use custom backgrounds. Choose:
\begin{itemize}
\item any two skill proficiencies 
\item any two tool proficiencies or languages 
\item one equipment package from any background
\item one background feature from any background
\end{itemize}

\subsection{Feats}
Feats (PH, 165) are allowed, except Lucky and Savage Attacker.

\subsubsection{New Prerequisites}

\paragraph{Dwarven Fortitude} (XGE, 74) You no longer need to be a Dwarf.

\paragraph{Grappler} (PH, 167) You no longer need a Strength score of 13 or higher.

\paragraph{Ritual Caster} (PH, 169) You no longer need an Intelligence or Wisdom score of 13 or higher.

\paragraph{Magic Initiate, Ritual Caster, Spell Sniper} (PH, 168-170) You need 11 points or more in the spellcasting ability you choose.



\label{flavour-feats}
\subsection{Flavour Feats}
At character levels 1, 4, 8, 12, 16 and 19, you can opt to take one flavour feat from the table in the Appendix (\pg{flavour-feats-table}). Ignore ability score increases from flavour feats.

Flavour feats are not strong, but have good roleplaying potential and provide situational benefits.











\chapter{Gameplay Options}
You can opt to use any number of these to modify your own gameplay.

\subsection{Proficiency Dice}
You can opt to use proficiency dice instead of a flat proficiency bonus (DMG, 263).

\subsection{Hit Dice}
This option gives you more precise control over healing from rests, but may provide less healing overall.

\subsubsection{Levelling}
You gain 2 hit dice per class level, instead of 1.

\subsubsection{Long Rests}
Long rests no longer automatically restore hit points. As a long rest ends, you can spend hit dice like in a short rest. Then, recover half your expended hit dice, rounded up.

\subsection{Encumbrance}
You can opt to track encumbrance as a self-imposed challenge. 

\begin{DndTable}[header=Encumbrance]{rX}
\textbf{Carried weight} & \textbf{Effect} \\
$\leq$ 5 $\times$ Strength & No penalty \\
$>$ 5 $\times$ Strength & Cannot travel at a fast pace \\
$>$ 10 $\times$ Strength & All speeds -10ft. \\
$>$ 15 $\times$ Strength & Physically impossible \\
\end{DndTable}


\subsection{Diagonal Grid Movement}
You can choose to move more realistically on a grid. Every second diagonal on a turn costs double (DMG, 252). Additionally, you have width while moving diagonally. 

\begin{DndComment}{Example}
Because you have width, if you step northeast you will also pass through the north and east grid squares. Therefore, you can't step diagonally if either of those squares are blocked, and you provoke opportunity attacks from both squares.
\end{DndComment}

\label{counterspell}
\subsection{Counterspell}
By the core rules, the spell "Counterspell" no longer exists. If you are a bard, or a spellcaster who originally had access to it, you can opt to gain this feature when you gain 3rd-level spells:

\subsubsection{Counterspell}
If you perceive a spell being cast, you automatically identify it, and can attempt to use your own spell to counter it, using your reaction and expending a spell slot. 

Your counterspell is a spell from the same school as the triggering spell, which you know or have prepared. You cast your counterspell while adjusting it to disrupt the triggering spell instead of producing its usual effect. Because higher-levelled spells use techniques not found in simpler spells, your counterspell's base level cannot be lower than the triggering spell's base level. 

Make a spellcasting ability check against DC 10 + the triggering spell's level. You gain a bonus to this check equal to the level of the spell slot you used. Note that this check is usually made without proficiency.

If your counterspell is the same spell as the triggering spell, you gain advantage on this check, as less adjustment is needed. If your counterspell is cast at a lower level than the triggering spell, you have disadvantage on this check.

\subsection{Critical Hits}
Whenever you make a critical hit, choose one of the following:
\begin{itemize}
\item maximise the dice (no uncertainty)
\item roll the dice twice (default)
\item roll once and double (gambling for a high roll)
\end{itemize}

\label{gameplay-bonus-action}
\subsection{Action: Gain a Bonus Action}
You can exchange an action for a bonus action on your turn, but you cannot use bonus actions from the same feature more than once in a turn.

\begin{DndComment}{Example}
You can use your action to gain a second bonus action, then cast Healing Word with one bonus action and use your remaining bonus action to attack with a Spiritual Weapon. 

However, you may not attack twice with a Spiritual Weapon because this uses two bonus actions from the same spell, even if you attack with two separate castings of the spell, or if one bonus action was used to attack with an existing Spiritual Weapon while the other was used to cast a new Spiritual Weapon.
\end{DndComment}

\subsection{Combat Options}
Besides the standard options like Dash and Shove, you can use Disarm, Shove Aside, Climb Onto, Overrun and Tumble (DMG, 271-272). 

For your convenience, a reference list of all combat options is provided in the Appendix \see{combat-options}.















\chapter{Clarifications}

This section covers situations which are unlikely to arise, but are still important enough to state clearly.

\subsection{Feat Preclusions}
You cannot be a Fighter (Battlemaster) and have the Martial Adept feat.

You cannot have levels in the same class in which you have the Magic Initiate feat.

If such a conflict arises, you can change the offending feat.

\subsection{Spellcasting Feats}

\subsubsection{Minimum Level}
Before you can cast a spell granted by a feat, you must first reach the level at which a full caster can cast that level of spell.

\subsubsection{Learning Spells}
Ignore any wording in feats that says you "learn" a spell.

\begin{DndComment}{Example}
If you take the Magic Initiate feat, you can cast the chosen 1st level spell once per long rest, but you do not learn that spell, and therefore cannot cast it using spell slots.
\end{DndComment}











\chapter{Appendix}

\section{Sources}
\label{sources}
If you don't have the sourcebooks, you can get them through these links. 

Please report broken links.

\begin{DndTable}[header=Sourcebooks (hyperlinked)]{l}
    \href{https://thetrove.is/Books/Dungeons\%20\&\%20Dragons/5th\%20Edition\%20(5e)/Core/Dungeon\%20Master\%27s\%20Guide.pdf}{Dungeon Master's Guide (DMG)} \\
    \href{https://thetrove.is/Books/Dungeons\%20\&\%20Dragons/5th\%20Edition\%20(5e)/Core/Eberron\%20-Rising\%20from\%20the\%20Last\%20War.pdf}{Eberron - Rising from the Last War (ERLW)} \\
    \href{https://media.wizards.com/2015/downloads/dnd/EE_PlayersCompanion.pdf}{Elemental Evil Player's Companion (EEPC)} \\
    \href{https://thetrove.is/Books/Dungeons\%20\&\%20Dragons/5th\%20Edition\%20(5e)/Core/Explorer\%27s\%20Guide\%20to\%20Wildemount.pdf}{Explorer's Guide to Wildemount (EGW)} \\
    \href{https://thetrove.is/Books/Dungeons\%20\&\%20Dragons/5th\%20Edition\%20(5e)/Core/Guildmasters\%27\%20Guide\%20to\%20Ravnica.pdf}{Guildmaster's Guide to Ravnica (GGR)} \\
    \href{https://thetrove.is/Books/Dungeons\%20\&\%20Dragons/5th\%20Edition\%20(5e)/Core/Monster\%20Manual\%20\%5B11th\%20Print\%5D.pdf}{Monster Manual (MM)} \\
    \href{https://thetrove.is/Books/Dungeons\%20\&\%20Dragons/5th\%20Edition\%20(5e)/Core/Mordenkainen\%27s\%20Tome\%20of\%20Foes.pdf}{Mordenkainen's Tome of Foes (MTF)} \\
    \href{https://thetrove.is/Books/Dungeons\%20\&\%20Dragons/5th\%20Edition\%20(5e)/Core/Player\%27s\%20Handbook\%20\%5B10th\%20Print\%5D.pdf}{Player's Handbook (PH)} \\
    \href{https://thetrove.is/Books/Dungeons\%20\&\%20Dragons/5th\%20Edition\%20(5e)/Core/Sword\%20Coast\%20Adventurer\%27s\%20Guide.pdf}{Sword Coast Adventurer's Guide (SCAG)} \\
    \href{https://thetrove.is/Books/Dungeons\%20\&\%20Dragons/5th\%20Edition\%20(5e)/Core/Volo\%27s\%20Guide\%20to\%20Monsters.pdf}{Volo's Guide to Monsters (VGM)} \\
    \href{https://thetrove.is/Books/Dungeons\%20\&\%20Dragons/5th\%20Edition\%20(5e)/Core/Xanathar\%27s\%20Guide\%20to\%20Everything.pdf}{Xanathar's Guide to Everything (XGE)} \\
\end{DndTable}

\label{combat-options}
\begin{DndTable}[header=Standard Combat Options]{llX}
\textbf{Cost} & \textbf{Option} & \textbf{Brief description} \\
Action & Dash & Extra movement. \\
Action & Disengage & Avoid opportunity attacks. \\
Action & Dodge & Advantage on attacks and Dex saves until next turn. \\
Action & Help & Give ally advantage on next check or attack. \\
Action & Hide & - \\
Action & Ready & Hold any other action until triggered. \\
Action/Bonus & Improvise & Do something nonstandard, at DM's discretion. \\
Action/Bonus & Search & Attempt to find a hidden object or creature. \\
1 attack & Grapple & Attempt  to grab opponent. (Ath vs. Ath/Acr) \\
1 attack & Shove & Attempt to push away or knock prone. (Ath vs. Ath/Acr)\\
\end{DndTable}

\begin{DndTable}[header=Advanced Combat Options]{llX}
\textbf{Cost} & \textbf{Option} & \textbf{Brief description} \\
Action & Gain B.A. & Gain another bonus action. \\
1 attack & Disarm & Attempt to knock weapon or item from opponent's hands. (Attack vs. Ath/Acr) \\
1 attack & Shove Aside & Attempt to push opponent to the side. (Ath[disadv] vs. Ath/Acr) \\
Half speed & Climb Onto & Attempt to climb onto bigger creature, gaining advantage on attacks against it. (Ath/Acr vs. Acr) \\
Free & Overrun & Attempt to charge through opponent's space. (Ath vs. Ath/Acr) \\
Free & Tumble & Attempt to slip through opponent's space. (Acr vs. Ath/Acr) \\
\end{DndTable}




\section{Official Races}
\label{official-races}
\begin{DndTable}[]{lXc}
    \textbf{Race} & \textbf{Subrace(s)} & \textbf{Source} \\
    Aarakocra & - & EEPC, 5 \\
    Aasimar & Fallen, Protector, Scourge & VGM, 104 \\
    Bugbear & - & ERLW, 25 \\
    Centaur & - & GGR, 15 \\
    Changeling & - & ERLW, 18 \\
    Dragonborn & - & PH, 34 \\
    Dwarf & Hill, Mountain & PH, 20 \\
      & Duergar & SCAG, 104 \\
      & Mark of Warding & ERLW, 51 \\
    Elf & Dark/Drow, High, Wood & PH, 23 \\
      & Eladrin, Sea, Shadar-kai & MTF, 62-63 \\
      & Mark of Shadow & ERLW, 49 \\
    Firbolg & - & VGM, 107 \\
    Genasi & Air, Earth, Fire, Water & EEPC, 9 \\
    Gith & Githyanki, Githzerai & MTF, 96 \\
    Gnome & Forest, Rock & PH, 36 \\
      & Deep & SCAG, 115 \\
      & Mark of Scribing & ERLW, 47 \\
    Goblin & - & ERLW, 26 \\
    Goliath & - & VGM, 109 \\
    Half-Elf & Standard & PH, 39 \\
      & Specific elven descent & SCAG, 116 \\
      & Marks of Detection, Storm & ERLW, 40,50 \\
    Half-Orc & - & PH, 41 \\
      & Mark of Finding & ERLW, 41 \\
    Halfling & Lightfoot, Stout & PH, 28 \\
      & Ghostwise & SCAG, 110 \\
      & Marks of Healing, Hospitality & ERLW, 43-44 \\
    Hobgoblin & - & ERLW, 26 \\
    Human & - & PH, 31 \\
      & Marks of Finding, Handling, Making, Passage, Sentinel & ERLW, 41-48 \\
    Kalashtar & - & ERLW, 30 \\
    Kenku & - & VGM, 111 \\
    Kobold & - & VGM, 119 \\
    Lizardfolk & - & VGM, 113 \\
    Loxodon & - & GGR, 18 \\
    Minotaur & - & GGR, 19 \\
    Orc & - & ERLW, 32 \\
    Shifter & Beasthide, Longtooth, Swiftstride, Wildhunt & ERLW, 33-34 \\
    Simic Hybrid & - & GGR, 20 \\
    Tabaxi & - & VGM, 115 \\
    Tiefling & Standard (aka Asmodeus) & PH, 43 \\
      & Devil's Tongue, Feral, Hellfire & SCAG, 118 \\
      & Winged & SCAG, 118 \\
      & Baalzebul, Dispater, Fierna, Glasya, Levistus, Mammon, Mephistopheles, Zariel & MTF, 21-23 \\
    Triton & - & VGM, 117 \\
    Vedalken & - & GGR, 21 \\
    Warforged & - & ERLW, 36 \\
    Yuan-ti & - & VGM, 120 \\
\end{DndTable}

\onecolumn
\section{Flavour Feats}
\label{flavour-feats-table}

You can take a single flavour feat, or a matched pair of feats (e.g. 3A and 3B) from the Flavour Feats (Paired) table. Such a pair counts as a single flavour feat. ASIs do not apply when feats are chosen as flavour feats, but are included for reference in case you choose these feats as normal feats.

\begin{DndTable}[header=Flavour Feats (Single)]{llll}
    \textbf{Feat} & \textbf{Prerequisite} & \textbf{ASI} & \textbf{Benefit} \\
    Actor & - & CHA & Mimic people and sounds. \\
    Bountiful Luck & Halfling & - & Give Lucky to allies. \\
    Charger & - & - & When Dashing, also attack or shove. \\
    Drow High Magic & Elf (Drow) & - & Detect magic, levitate, and dispel magic. \\
    Dungeon Delver & - & - & Find secret doors and resist traps. \\
    Elemental Adept & Spells & - & Strengthen spells of one element. \\
    Fade Away & Gnome & DEX/INT & Turn invisible when damaged. \\
    Fey Teleportation & Elf (High) & INT/CHA & Learn Sylvan and misty step. \\
    Healer & - & - & Nonmagical healing. \\
    Keen Mind & - & INT & Photographic memory. \\
    Mage Slayer & - & - & Harass casters in melee. \\
    Magic Initiate & Casting ability 11+ & - & Two cantrips and one spell from any class. \\
    Martial Adept & - & - & Combat manoeuvres. \\
    Mounted Combatant & - & - & More accuracy and protect your mount. \\
    Orcish Fury & Half-Orc & STR/CON & Burst damage and revenge when knocked out. \\
    Prodigy & Human / Half-Elf / Half-Orc & - & One each of language, skill, tool, expertise. \\
    Ritual Caster & Casting ability 11+ & - & Ritual spells from one class. \\
    Second Chance & Halfling & DEX/CON/CHA & Dodge one attack per combat. \\
    Shield Master & - & - & Better DEX saves, shove with shield. \\
    Skilled & - & - & 3 skills or tools. \\
    Skulker & Dexterity 13+ & - & More hiding places, see better in dim light. \\
    Spell Sniper & Spells, Casting ability 11+ & - & Doubles spell attack range, learn one cantrip. \\
    Squat Nimbleness & Dwarf or Small race & STR/DEX & Move faster, learn Ath/Acr, escape grapples.\\
    Svirfneblin Magic & Gnome (Deep) & - & Learn a variety of sneak and illusion magic. \\
    Wood Elf Magic & Elf (Wood) & - & Druid cantrip, longstrider, pass without trace. \\
\end{DndTable}

\begin{DndTable}[header=Flavour Feats (Paired)]{lllll}
    \textbf{Pair} & \textbf{Feat} & \textbf{Prerequisite} & \textbf{ASI} & \textbf{Benefit} \\
    \hline
    1A & Athlete & - & STR/DEX & Easily climb, jump and stand from prone. \\
    1B & Weapon Master & - & STR/DEX & Four weapon proficiencies. \\
    \hline
    2A & Dragon Fear & Dragonborn & STR/CON/CHA & Use breath weapon to frighten. \\
    2B & Dragon Hide & Dragonborn & STR/CON/CHA & Natural armor and claws. \\
    \hline
    3A & Durable & - & CON & More reliable healing from hit dice. \\
    3B & Dwarven Fortitude & \textbf{none} & CON & Heal while Dodging. \\
    \hline
    4A & Flames of Phlegethos & Tiefling & INT/CHA & Somewhat improve fire spells. \\
    4B & Infernal Constitution & Tiefling & CON & Resist cold and poison. \\
    \hline
    5A & Grappler & - & - & Advantage to hit grapplee, or pin them. \\
    5B & Tavern Brawler & - & STR/CON & Fight and grapple, without weapons. \\
    \hline
    6A & Linguist & - & INT & Learn 3 languages, and create ciphers. \\
    6B & Observant & - & INT/WIS & Read lips, better passive alertness. \\
\end{DndTable}

\twocolumn

\end{document}